% !TeX encoding = UTF-8
% !TeX spellcheck = en_US


\documentclass[12pt,a4paper]{article}



\usepackage[latin1]{inputenc}
\usepackage[english]{babel}
\usepackage{indentfirst}
\usepackage{amsmath}
\usepackage{amsfonts}
\usepackage{amssymb}
\usepackage[left=3cm,right=3cm,bottom=3.3cm]{geometry}
\usepackage[colorlinks=true,linkcolor=blue,citecolor=blue]{hyperref}




\title{Electron-phonon matrix elements for non-local pseudopotentials}




\allowdisplaybreaks
\begin{document}

\renewcommand{\vec}{\mathbf}


\maketitle





\section{Unit cell}

The unit cell of a crystal is defined by the vectors $\vec{a}_1$, $\vec{a}_2$ and $\vec{a}_3$.

It's volume is $\Omega=\vec{a}_1 \cdot (\vec{a}_2 \times \vec{a}_3)$.

The vectors $\vec{a}_1$, $\vec{a}_2$ and $\vec{a}_3$ define the Bravais lattice $\vec{R}$ of the crystal.

On the other hand, the reciprocal space is defined by the vectors $\vec{b}_1$, $\vec{b}_2$ and $\vec{b}_3$, where $\vec{a}_i \cdot \vec{b}_j = 2\pi\delta_{ij}$.

The vectors $\vec{b}_1$, $\vec{b}_2$ eta $\vec{b}_3$ define the Bravais lattice $\vec{G}$ of the reciprocal space.





\section{Wave functions}

\subsection*{Bloch's theorem}

\begin{align}
\psi_{n\vec{k}}(\vec{r})
&=
e^{i\vec{k}\cdot\vec{r}}
u_{n\vec{k}}(\vec{r})
,\end{align}
where $u_{n\vec{k}}(\vec{r}+\vec{R}) = u_{n\vec{k}}(\vec{r})$, for a Bravais lattice vector $\vec{R}$.


\subsection*{Born-von Karman boundary conditions}

The Born-von Karman boundary conditions are imposed in a super-cell defined by the vectors $\vec{A}_1=N_1\vec{a}_1$, $\vec{A}_2=N_2\vec{a}_2$ and $\vec{A}_3=N_3\vec{a}_3$:
\begin{align}
\psi_{n\vec{k}}(\vec{r}+\vec{R}_{SC})
&=
\psi_{n\vec{k}}(\vec{r})
,\end{align}
where $\vec{R}_{SC}=\sum_{i=1}^{3} n_i \vec{A}_i$ is a Bravais lattice vector of the super-cell.

Due to the Bloch-en theorem:
\begin{align}
\psi_{n\vec{k}}(\vec{r}+\vec{R}_{SC})
&=
e^{i\vec{k}\cdot(\vec{r}+\vec{R}_{SC})}
u_{n\vec{k}}(\vec{r}+\vec{R}_{SC})
\\
&=
e^{i\vec{k}\cdot(\vec{r}+\vec{R}_{SC})}
u_{n\vec{k}}(\vec{r})
\\
&=
e^{i\vec{k}\cdot\vec{r}}
e^{i\vec{k}\cdot\vec{R}_{SC}}
u_{n\vec{k}}(\vec{r})
\\
&=
e^{i\vec{k}\cdot\vec{R}_{SC}}
\psi_{n\vec{k}}(\vec{r})
.\end{align}
Then, in order to satisfy Born-von Karman boundary conditions:
\begin{align}
\vec{k}\cdot\vec{R}_{SC}
=
\sum_{i=1}^{3}
\vec{k}_i n_i \vec{A}_i
=
\sum_{i=1}^{3}
\vec{k}_i n_i N_i \vec{a}_i
=
2\pi n
,\end{align}
and therefore,
\begin{align}\label{eq:Born-von_Karman-k-vectors}
\vec{k}
=
\sum_{i=1}^{3}
\frac{m_i}{N_i}\vec{b}_i
,\quad
m_i \in \mathbb{N}
.\end{align}

\subsection*{Fourier transform}

Due to the Born-von Karman boundary conditions the wave function is periodic in the super-cell, and therefore, the Fourier expansion of the wave function can be used:
\begin{align}\label{eq:wave-function-from-G-to-r}
\psi_{n\vec{k}}(\vec{r})
&=
\sum_{\vec{G}_{SC}}
\tilde{\psi}_{n\vec{k}}(\vec{G}_{SC})
e^{i\vec{G}_{SC}\cdot\vec{r}}
,\end{align}
where $\vec{G}_{SC}$ are the reciprocal Bravais lattice vectors of the super-cell, which satisfy the condition of eq.~\eqref{eq:Born-von_Karman-k-vectors}, and
\begin{align}\label{eq:wave-function-from-r-to-G}
\tilde{\psi}_{n\vec{k}}(\vec{G}_{SC})
&=
\frac{1}{\Omega_{SC}}
\int_{\Omega_{SC}}
\psi_{n\vec{k}}(\vec{r})
e^{-i\vec{G}_{SC}\cdot\vec{r}}
d\vec{r}
,\end{align}
where $\Omega_{SC}=N_{SC}\Omega$, with $N_{SC} = N_1 \times N_2 \times N_3$.

On the other hand, $u_{n\vec{k}}(\vec{r})$ is periodic in the unit cell, and therefore it's Fourier expansion:
\begin{align}\label{eq:periodic-wave-function-from-G-to-r}
u_{n\vec{k}}(\vec{r})
&=
\sum_{\vec{G}}
\tilde{u}_{n\vec{k}}(\vec{G})
e^{i\vec{G}\cdot\vec{r}}
,\end{align}
where
\begin{align}\label{eq:periodic-wave-function-from-r-to-G}
\tilde{u}_{n\vec{k}}(\vec{G})
&=
\frac{1}{\Omega}
\int_{\Omega}
u_{n\vec{k}}(\vec{r})
e^{-i\vec{G}\cdot\vec{r}}
d\vec{r}
.\end{align}

Using Bloch's theorem and the Fourier expansion of the periodic part of the wave function:
\begin{align}
\psi_{n\vec{k}}(\vec{r})
&=
e^{i\vec{k}\cdot\vec{r}}
u_{n\vec{k}}(\vec{r})
\\
&=
e^{i\vec{k}\cdot\vec{r}}
\sum_{\vec{G}}
\tilde{u}_{n\vec{k}}(\vec{G})
e^{i\vec{G}\cdot\vec{r}}
\\
&=
\sum_{\vec{G}}
\tilde{u}_{n\vec{k}}(\vec{G})
e^{i(\vec{k}+\vec{G})\cdot\vec{r}}
.\end{align}
Thus, due to the crystal periodicity and Bolch's theorem, the Fourier expansion of the wave function $\psi_{n\vec{k}}$ only has non-zero $\tilde{\psi}_{n\vec{k}}(\vec{G}_{SC})$ components that differ form $\vec{k}$ by a reciplocal space lattice vector $\vec{G}$. And, in fact,
\begin{align}
\tilde{\psi}_{n\vec{k}}(\vec{k}+\vec{G})
&=
\tilde{u}_{n\vec{k}}(\vec{G})
.\end{align}

This last conclusion can be checked by analyzing the $\tilde{\psi}_{n\vec{k}}(\vec{k}+\vec{G})$ components from eq.~\eqref{eq:wave-function-from-r-to-G}:
\begin{align}
\nonumber
\tilde{\psi}_{n\vec{k}}(\vec{k}+\vec{G})
&=
\frac{1}{\Omega_{SC}}
\int_{\Omega_{SC}}
\psi_{n\vec{k}}(\vec{r})
e^{-i(\vec{k}+\vec{G})\cdot\vec{r}}
d\vec{r}
\\
&=
\frac{1}{\Omega_{SC}}
\int_{\Omega_{SC}}
u_{n\vec{k}}(\vec{r})
e^{-i\vec{G}\cdot\vec{r}}
d\vec{r}
.\end{align}
And taking into account that both $u_{n\vec{k}}(\vec{r})$ and $e^{-i\vec{G}\cdot\vec{r}}$ are periodic in the unit cell:
\begin{align}
\int_{\Omega_{SC}}
u_{n\vec{k}}(\vec{r})
e^{-i\vec{G}\cdot\vec{r}}
d\vec{r}
&=
N_{SC}
\int_{\Omega}
u_{n\vec{k}}(\vec{r})
e^{-i\vec{G}\cdot\vec{r}}
d\vec{r}
.\end{align}
And therefore:
\begin{align}
\tilde{\psi}_{n\vec{k}}(\vec{k}+\vec{G})
&=
\frac{1}{\Omega_{SC}}
N_{SC}
\int_{\Omega}
u_{n\vec{k}}(\vec{r})
e^{-i\vec{G}\cdot\vec{r}}
d\vec{r}
\\
&=
\frac{1}{\Omega}
\int_{\Omega}
u_{n\vec{k}}(\vec{r})
e^{-i\vec{G}\cdot\vec{r}}
d\vec{r}
\\
&=
\tilde{u}_{n\vec{k}}(\vec{G})
.\end{align}







\subsection*{k+G}

While $u_{n\vec{k}}$ are periodic in real space, the wave functions $\psi_{n\vec{k}}$ are periodic in reciprocal space:
\begin{align}
\psi_{n\vec{k}+\vec{G}}(\vec{r})
&=
\psi_{n\vec{k}}(\vec{r})
.\end{align}
Therefore,
\begin{align}
\psi_{n\vec{k}+\vec{G}}(\vec{r})
&=
u_{n\vec{k}+\vec{G}}(\vec{r})
e^{i(\vec{k}+\vec{G})\cdot\vec{r}}
=
u_{n\vec{k}}(\vec{r})
e^{-i\vec{G}\cdot\vec{r}}
\end{align}
and thus:
\begin{align}
u_{n\vec{k}+\vec{G}}(\vec{r})
&=
u_{n\vec{k}}(\vec{r})
e^{-i\vec{G}\cdot\vec{r}}
.\end{align}

And in reciprocal space:
\begin{align}
\tilde{u}_{n\vec{k}+\vec{G}}(\vec{G}')
&=
\frac{1}{\Omega}\int_{\Omega}
u_{n\vec{k}+\vec{G}}(\vec{r})
e^{-i\vec{G}'\cdot\vec{r}}
d\vec{r}
\\
&=
\frac{1}{\Omega}\int_{\Omega}
u_{n\vec{k}}(\vec{r})
e^{-i(\vec{G}+\vec{G}')\cdot\vec{r}}
d\vec{r}
\\
&=
\tilde{u}_{n\vec{k}}(\vec{G}+\vec{G}')
.\end{align}



\subsection*{Orthogonality}

\begin{align}
\langle\psi_{n\vec{k}}|\psi_{n'\vec{k}}\rangle
&=
\frac{1}{\Omega_{SC}}
\int_{\Omega_{SC}}
\psi_{n\vec{k}}^*(\vec{r})
\psi_{n'\vec{k}}(\vec{r})
d\vec{r}
\\
&=
\frac{1}{\Omega_{SC}}
\int_{\Omega_{SC}}
e^{-i\vec{k}\cdot\vec{r}} u_{n\vec{k}}^*(\vec{r})
e^{i\vec{k}\cdot\vec{r}} u_{n'\vec{k}}(\vec{r})
d\vec{r}
\\
&=
\frac{1}{\Omega}
\int_{\Omega}
u_{n\vec{k}}^*(\vec{r})
u_{n'\vec{k}}(\vec{r})
d\vec{r}
\\
&=
\delta_{n,n'}
.\end{align}




\section{Pseudo-potential}

\subsection*{Local part}

The local part of the pseudo-potential is:
\begin{align}
V_{loc}(\vec{r})
&=
\sum_{s,\vec{R}}
v_s(\vec{r}-\boldsymbol{\tau}_{s}-\vec{R})
,\end{align}
where $\boldsymbol{\tau}_{s}$ is the equilibrium position of the atom $s$ of the unit cell.

Since the local potential is periodic in the unit cell, $V_{loc}(\vec{r}+\vec{R})=V_{loc}(\vec{r})$, it can be expanded using a Fourier series:
\begin{align}
V_{loc}(\vec{r})
&=
\sum_{\vec{G}}
\tilde{V}_{loc}(\vec{G})
e^{i\vec{G}\cdot\vec{r}}
,\end{align}
where
\begin{align}
\tilde{V}_{loc}(\vec{G})
&=
\frac{1}{\Omega}
\int_{\Omega}
V_{loc}(\vec{r})
e^{-i\vec{G}\cdot\vec{r}}
d\vec{r}
\\
&=
\frac{1}{\Omega}
\sum_{s,\vec{R}}
\int_{\Omega}
v_s(\vec{r}-\boldsymbol{\tau}-\vec{R})
e^{-i\vec{G}\cdot\vec{r}}
d\vec{r}
,\end{align}
making the change of variables $\vec{r}'=\vec{r}-\vec{R}$, the limits of the integral will change:
\begin{align}
\tilde{V}_{loc}(\vec{G})
&=
\frac{1}{\Omega}
\sum_{s,\vec{R}}
\int_{\Omega(-\vec{R})}
v_s(\vec{r}'-\boldsymbol{\tau}_{s})
e^{-i\vec{G}\cdot\vec{r}'}
d\vec{r}'
,\end{align}
where $\Omega(\vec{R})$ represents the unit cell with origin in the Bravais lattice vector $\vec{R}$. Therefore, $\sum_\vec{R}\int_{\Omega(-\vec{R})} \rightarrow \int_{\text{full space}}$:
\begin{align}
\tilde{V}_{loc}(\vec{G})
&=
\frac{1}{\Omega}
\sum_{s}
\int_{\text{full space}}
v_s(\vec{r}'-\boldsymbol{\tau}_{s})
e^{-i\vec{G}\cdot\vec{r}'}
d\vec{r}'
\\
&=
\frac{1}{\Omega}
\sum_{s}
e^{-i\vec{G}\cdot\boldsymbol{\tau}_{s}}
\int_{\text{full space}}
v_s(\vec{r}'')
e^{-i\vec{G}\cdot\vec{r}''}
d\vec{r}''
\\
&=
\sum_{s}
\tilde{v}_s(\vec{G})
,\end{align}
with
\begin{align}
\tilde{v}_s(\vec{G})
=
\frac{1}{\Omega}
e^{-i\vec{G}\cdot\boldsymbol{\tau}_{s}}
\int_{\text{full space}}
v_s(\vec{r}'')
e^{-i\vec{G}\cdot\vec{r}''}
d\vec{r}''
.\end{align}



\subsection*{Non-local part}

The non-local part of the pseudo-potential is:
\begin{align}
V_{NL}(\vec{r}',\vec{r}'')
&=
\sum_{s,i,i',\vec{R}}
D_{si,si'}
\beta_{s,i}(\vec{r}'-\boldsymbol{\tau}_{s}-\vec{R})
\beta_{s,i'}^*(\vec{r}''-\boldsymbol{\tau}_{s}-\vec{R})
.\end{align}

As for the local part, the non-local part is also periodic in the unit cell,
\begin{align}
V_{NL}(\vec{r}'+\vec{R},\vec{r}''+\vec{R}')=V_{NL}(\vec{r}',\vec{r}'')
,\end{align}
and can be expanded in a Fourier series:
\begin{align}
V_{NL}(\vec{r}',\vec{r}'')
&=
\sum_{\vec{G}',\vec{G}''}
\tilde{V}_{NL}(\vec{G}',\vec{G}'')
e^{i\vec{G}'\cdot\vec{r}'}
e^{-i\vec{G}''\cdot\vec{r}''}
,\end{align}
where
\begin{align}
\tilde{V}_{NL}(\vec{G}',\vec{G}'')
&=
\frac{1}{\Omega^2}
\int_{\Omega}
\int_{\Omega}
V_{NL}(\vec{r}',\vec{r}'')
e^{-i\vec{G}'\cdot\vec{r}'}
e^{i\vec{G}''\cdot\vec{r}''}
d\vec{r}'
d\vec{r}''
\\
&=
\frac{1}{\Omega^2}
\sum_{s,i,i',\vec{R}}
D_{si,si'}
\int_{\Omega}
\beta_{s,i}(\vec{r}'-\boldsymbol{\tau}_{s}-\vec{R})
e^{-i\vec{G}'\cdot\vec{r}'}
d\vec{r}'
\int_{\Omega}
\beta_{s,i'}^*(\vec{r}''-\boldsymbol{\tau}_{s}-\vec{R})
e^{i\vec{G}''\cdot\vec{r}''}
d\vec{r}''
\\
&=
\frac{1}{\Omega^2}
\sum_{s,i,i',\vec{R}}
D_{si,si'}
\int_{\Omega(\-\vec{R})}
\beta_{s,i}(\vec{r}'_1-\boldsymbol{\tau}_{s})
e^{-i\vec{G}'\cdot\vec{r}'_1}
d\vec{r}'_1
\int_{\Omega(\-\vec{R})}
\beta_{s,i'}^*(\vec{r}'_2-\boldsymbol{\tau}_{s})
e^{i\vec{G}''\cdot\vec{r}'_2}
d\vec{r}'_2
\\
&=
\frac{1}{\Omega^2}
\sum_{s,i,i'}
D_{si,si'}
\int_{\text{full space}}
\beta_{s,i}(\vec{r}'_1-\boldsymbol{\tau}_{s})
e^{-i\vec{G}'\cdot\vec{r}'_1}
d\vec{r}'_1
\int_{\text{full space}}
\beta_{s,i'}^*(\vec{r}'_2-\boldsymbol{\tau}_{s})
e^{i\vec{G}''\cdot\vec{r}'_2}
d\vec{r}'_2
\\
&=
\frac{1}{\Omega^2}
\sum_{s,i,i'}
D_{si,si'}
e^{-i(\vec{G}'-\vec{G}'')\cdot\boldsymbol{\tau}_{s}}
\int_{\text{full space}}
\beta_{s,i}(\vec{r}''_1)
e^{-i\vec{G}'\cdot\vec{r}''_1}
d\vec{r}''_1
\int_{\text{full space}}
\beta_{s,i'}^*(\vec{r}''_2)
e^{i\vec{G}''\cdot\vec{r}''_2}
d\vec{r}''_2
\\\nonumber
&=
\frac{1}{\Omega^2}
\sum_{s,i,i'}
D_{si,si'}
\left[
e^{-i\vec{G}'\cdot\boldsymbol{\tau}_{s}}
\int_{\text{full space}}
\beta_{s,i}(\vec{r}''_1)
e^{-i\vec{G}'\cdot\vec{r}''_1}
d\vec{r}''_1
\right]
\\
&\hspace{2.4cm}
\times
\left[
e^{-i\vec{G}''\cdot\boldsymbol{\tau}_{s}}
\int_{\text{full space}}
\beta_{s,i'}(\vec{r}''_2)
e^{-i\vec{G}''\cdot\vec{r}''_2}
d\vec{r}''_2
\right]^*
\\
&=
\sum_{s,i,i'}
D_{si,si'}
\tilde{\beta}_{s,i}(\vec{G}')
\tilde{\beta}_{s,i'}^*(\vec{G}'')
,\end{align}
with
\begin{align}
\tilde{\beta}_{s,i}(\vec{G})
&=
\frac{1}{\Omega}
e^{-i\vec{G}\cdot\boldsymbol{\tau}_{s}}
\int_{\text{full space}}
\beta_{s,i}(\vec{r})
e^{-i\vec{G}\cdot\vec{r}}
d\vec{r}
.\end{align}






\section{Electron-phonon matrix elements}

\subsection*{Local part}

The atomic positions for a phonon with wave vector $\vec{q}$ are
\begin{align}
\boldsymbol{\tau}_{s}(\vec{R})
&=
\boldsymbol{\tau}_{s}
+
\vec{R}
+
\vec{u}_{s,\vec{R}}
\\
&=
\boldsymbol{\tau}_{s}
+
\vec{R}
+
\vec{u}_{s,\vec{q}}
e^{i\vec{q}\cdot\vec{R}}
,\end{align}
where $\boldsymbol{\tau}_{s}$ is the equilibrium position of the atom $s$ of the unit cell, $\vec{R}$ a lattice vector, and $\vec{u}_{s,\vec{q}}$ the polarization vector of the phonon. Due to this displacement, the periodicity of the crystal is broken, however, using the Born-von Karman boundary conditions, the crystal will still be periodic in a Born-von Karman super-cell defined by the vectors $\vec{A}_1=N_1\vec{a}_1$, $\vec{A}_2=N_2\vec{a}_2$ and $\vec{A}_3=N_3\vec{a}_3$:
\begin{align}
\boldsymbol{\tau}_{s}(\vec{R}+\vec{R}_{SC})
&=
\boldsymbol{\tau}_{s}(\vec{R})
,\end{align}
for
\begin{align}\label{eq:Born-von_Karman-q-vectors}
\vec{q}
=
\sum_{i=1}^{3}
\frac{m_i}{N_i}\vec{b}_i
,\quad
m_i \in \mathbb{N}
.\end{align}
Therefore, the local part of the pseudo-potential can be expanded in a Fourier series:
\begin{align}\label{eq:V^q_loc}
V_{loc}^{\vec{q}}(\vec{r})
&=
\sum_{s,\vec{R}}
v_s(\vec{r}-\boldsymbol{\tau}_{s}-\vec{R}-\vec{u}_{s,\vec{q}}
e^{i\vec{q}\cdot\vec{R}})
\\
&=
\sum_{\vec{G}_{SC}}
\tilde{V}_{loc}^{\vec{q}}(\vec{G}_{SC})
e^{i\vec{G}_{SC}\cdot\vec{r}}
,\end{align}
where the $\vec{G}_{SC}$ vectors satisfy eq.~\eqref{eq:Born-von_Karman-q-vectors}, and
\begin{align}
\tilde{V}_{loc}^{\vec{q}}(\vec{G}_{SC})
&=
\frac{1}{\Omega_{SC}}
\int_{\Omega_{SC}}
V_{loc}^{\vec{q}}(\vec{r})
e^{-i\vec{G}_{SC}\cdot\vec{r}}
d\vec{r}
.\end{align}
Using $\vec{G}_{SC}=\vec{q}'+\vec{G}$, where $\vec{q}'\in{1BZ}$:
\begin{align}
V_{loc}^{\vec{q}}(\vec{r})
&=
\sum_{\vec{q}'\in\text{1BZ}}
\sum_{\vec{G}}
\tilde{V}_{loc}^{\vec{q}}(\vec{q}'+\vec{G})
e^{i(\vec{q}'+\vec{G})\cdot\vec{r}}
.\end{align}

Eq.~\eqref{eq:V^q_loc} uses a summation over the atoms within the unit cell of the crystal and over the lattice vectors $\vec{R}$ of the crystal. However, in this case, it is useful to make the summations with respect to the super-cell:
\begin{align}
V_{loc}^{\vec{q}}(\vec{r})
&=
\sum_{s'\in{UC}}
\sum_{\vec{R}}
v_{s'}(\vec{r}-\boldsymbol{\tau}_{s'}-\vec{R}-\vec{u}_{s',\vec{q}}
e^{i\vec{q}\cdot\vec{R}})
\\
&=
\sum_{S'\in{SC}}
\sum_{\vec{R}_{SC}}
v_{S'}(\vec{r}-\boldsymbol{\tau}_{S'}-\vec{R}_{SC}-\vec{u}_{S',\vec{q}})
,\end{align}
where $\vec{u}_{S',\vec{q}}=\vec{u}_{s',\vec{q}}
e^{i\vec{q}\cdot\vec{R}_{S',s'}}=\vec{u}_{s',\vec{q}}
e^{i\vec{q}\cdot(\boldsymbol{\tau}_{S'}-\boldsymbol{\tau}_{s'})}$.
Therefore:
\begin{align}
\tilde{V}_{loc}^{\vec{q}}(\vec{q}'+\vec{G})
&=
\frac{1}{\Omega_{SC}}
\int_{\Omega_{SC}}
V_{loc}^{\vec{q}}(\vec{r})
e^{-i(\vec{q}'+\vec{G})\cdot\vec{r}}
d\vec{r}
\\
&=
\frac{1}{\Omega_{SC}}
\sum_{S'\in{SC}}
\sum_{\vec{R}_{SC}}
\int_{\Omega_{SC}}
v_{S'}(
\vec{r}
-\boldsymbol{\tau}_{S'}
-\vec{R}_{SC}
-\vec{u}_{S',\vec{q}}
)
e^{-i(\vec{q}'+\vec{G})\cdot\vec{r}}
d\vec{r}
\\
&=
\frac{1}{\Omega_{SC}}
\sum_{S'\in{SC}}
\sum_{\vec{R}_{SC}}
\int_{\Omega_{SC}(-\vec{R}_{SC})}
v_{S'}(
\vec{r}'
-\boldsymbol{\tau}_{S'}
-\vec{u}_{S',\vec{q}}
)
e^{-i(\vec{q}'+\vec{G})\cdot\vec{r}'}
d\vec{r}'
\\
&=
\frac{1}{\Omega_{SC}}
\sum_{S'\in{SC}}
\int_{\text{full space}}
v_{S'}(
\vec{r}'
-\boldsymbol{\tau}_{S'}
-\vec{u}_{S',\vec{q}}
)
e^{-i(\vec{q}'+\vec{G})\cdot\vec{r}'}
d\vec{r}'
\\
&=
\frac{1}{\Omega_{SC}}
\sum_{S'\in{SC}}
e^{-i(\vec{q}'+\vec{G})\cdot(\boldsymbol{\tau}_{S'}+\vec{u}_{S',\vec{q}})}
\int_{\text{full space}}
v_{S'}(
\vec{r}''
)
e^{-i(\vec{q}'+\vec{G})\cdot\vec{r}''}
d\vec{r}''
\\
&=
\frac{1}{N_{SC}}
\sum_{S'\in{SC}}
e^{-i(\vec{q}'+\vec{G})\cdot\vec{u}_{S',\vec{q}}}
\tilde{v}_{S'}(\vec{q}'+\vec{G})
.\end{align}
Writing this expression again in terms of the unit cell indices,
\begin{align}
\sum_{S\in{SC}}
\rightarrow
\sum_{s\in{UC}}
\sum_{\vec{R}\in{SC}}
,\end{align}
where $\vec{R}\in{SC}$ are the $N_1\times N_2\times N_3$ lattice vectors that lie inside the super-cell. Therefore:
\begin{align}
\tilde{V}_{loc}^{\vec{q}}(\vec{q}'+\vec{G})
&=
\frac{1}{N_{SC}}
\sum_{s'\in{UC}}
\sum_{\vec{R}\in{SC}}
e^{-i\vec{q}'\cdot\vec{R}}
e^{-i(\vec{q}'+\vec{G})\cdot\vec{u}_{s',\vec{q}}e^{i\vec{q}\cdot\vec{R}}}
\tilde{v}_{s'}(\vec{q}'+\vec{G})
.\end{align}

Now, the first derivative of the $V_{loc}^{\vec{q}}$ potential with respect to the phonon polarization $\vec{u}_{s,\vec{q}}$ along the cartesian direction $\alpha$ around the atomic equilibrium positions:
\begin{align}
\frac{d V_{loc}^{\vec{q}}(\vec{r})}{d \mathrm{u}_{s,\vec{q}}^{\alpha}}
|_{\vec{u}=0}
&=
\sum_{\vec{q}'\in\text{1BZ}}
\sum_{\vec{G}}
\frac{d \tilde{V}_{loc}^{\vec{q}}(\vec{q}'+\vec{G})}{d \mathrm{u}_{s,\vec{q}}^{\alpha}}
|_{\vec{u}=0}
e^{i(\vec{q}'+\vec{G})\cdot\vec{r}}
,\end{align}
where,
\begin{align}
\nonumber
\frac{d \tilde{V}_{loc}^{\vec{q}}(\vec{q}'+\vec{G})}{d \mathrm{u}_{s,\vec{q}}^{\alpha}}
|_{\vec{u}=0}
&=
\frac{1}{N_{SC}}
\sum_{s'\in{UC}}
\sum_{\vec{R}\in{SC}}
e^{-i\vec{q}'\cdot\vec{R}}
\left[
\frac{d }{d \mathrm{u}_{s,\vec{q}}^{\alpha}}
e^{-i(\vec{q}'+\vec{G})\cdot\vec{u}_{s',\vec{q}}e^{i\vec{q}\cdot\vec{R}}}
\right]
|_{\vec{u}=0}
\\
&\hspace{3.1cm}
\times
\tilde{v}_{s'}(\vec{q}'+\vec{G})
\\\nonumber
&=
\frac{1}{N_{SC}}
\sum_{s'\in{UC}}
\sum_{\vec{R}\in{SC}}
e^{-i\vec{q}'\cdot\vec{R}}
\left[
-i(\mathrm{q}'_{\alpha}+\mathrm{G}_{\alpha})
e^{i\vec{q}\cdot\vec{R}}
\delta_{s,s'}
e^{-i(\vec{q}'+\vec{G})\cdot\vec{u}_{s,\vec{q}}e^{i\vec{q}\cdot\vec{R}}}
\right]
|_{\vec{u}=0}
\\
&\hspace{3.1cm}
\times
\tilde{v}_{s'}(\vec{q}'+\vec{G})
\\
&=
\frac{1}{N_{SC}}
\sum_{\vec{R}\in{SC}}
e^{-i\vec{q}'\cdot\vec{R}}
\left[
-i(\mathrm{q}'_{\alpha}+\mathrm{G}_{\alpha})
e^{i\vec{q}\cdot\vec{R}}
\right]
\tilde{v}_s(\vec{q}'+\vec{G})
\\
&=
-i(\mathrm{q}'_{\alpha}+\mathrm{G}_{\alpha})
\tilde{v}_s(\vec{q}'+\vec{G})
\frac{1}{N_{SC}}
\sum_{\vec{R}\in{SC}}
e^{-i(\vec{q}'-\vec{q})\cdot\vec{R}}
.\end{align}
It can be check that 
\begin{align}
\frac{1}{N_{SC}}
\sum_{\vec{R}\in{SC}}
e^{-i(\vec{q}'-\vec{q})\cdot\vec{R}}
&=
\delta_{\vec{q}',\vec{q}}
,\end{align}
and therefore:
\begin{align}
\partial_{s,\vec{q}}^{\alpha} V_{loc}(\vec{q}'+\vec{G})
\equiv
\frac{d \tilde{V}_{loc}^{\vec{q}}(\vec{q}'+\vec{G})}{d \mathrm{u}_{s,\vec{q}}^{\alpha}}
|_{\vec{u}=0}
&=
-i(\mathrm{q}_{\alpha}+\mathrm{G}_{\alpha})
\tilde{v}_s(\vec{q}+\vec{G})
\delta_{\vec{q}',\vec{q}}
,\end{align}
and
\begin{align}
\partial_{s,\vec{q}}^{\alpha} V_{loc}(\vec{r})
\equiv
\frac{d V_{loc}^{\vec{q}}(\vec{r})}{d \mathrm{u}_{s,\vec{q}}^{\alpha}}
|_{\vec{u}=0}
&=
\sum_{\vec{q}'\in\text{1BZ}}
\sum_{\vec{G}}
-i(\mathrm{q}_{\alpha}+\mathrm{G}_{\alpha})
\tilde{v}_s(\vec{q}+\vec{G})
e^{i(\vec{q}'+\vec{G})\cdot\vec{r}}
\delta_{\vec{q}',\vec{q}}
\\
&=
\sum_{\vec{G}}
-i(\mathrm{q}_{\alpha}+\mathrm{G}_{\alpha})
\tilde{v}_s(\vec{q}+\vec{G})
e^{i(\vec{q}+\vec{G})\cdot\vec{r}}
,\end{align}


Finally, the electron-phonon matrix element is:
\begin{align}
\langle\psi_{n\vec{k}+\vec{q}}|
\partial_{s,\vec{q}}^{\alpha} V_{loc}
|\psi_{m\vec{k}}\rangle
&=
\int
\int
\langle \psi_{n\vec{k}+\vec{q}} | \vec{r} \rangle
\langle \vec{r} | \partial_{s,\vec{q}}^{\alpha} V_{loc} | \vec{r}' \rangle
\langle \vec{r}' | \psi_{m\vec{k}} \rangle
d\vec{r}
d\vec{r}'
\\
&=
\int
\int
\psi_{n\vec{k}+\vec{q}}^*(\vec{r})
\partial_{s,\vec{q}}^{\alpha} V_{loc}(\vec{r})
\delta(\vec{r}-\vec{r}')
\psi_{m\vec{k}}(\vec{r}')
d\vec{r}
d\vec{r}'
\\\nonumber
&=
\int
\left[
\sum_{\vec{G}}
\tilde{u}_{n\vec{k}+\vec{q}}(\vec{G})
e^{i(\vec{k}+\vec{q}+\vec{G})\cdot\vec{r}}
\right]^*
\left[
\sum_{\vec{G}'}
-i(\mathrm{q}_{\alpha}+\mathrm{G}'_{\alpha})
\tilde{v}_s(\vec{q}+\vec{G}')
e^{i(\vec{q}+\vec{G}')\cdot\vec{r}}
\right]
\\
&
\phantom{=\int}
\times
\left[
\sum_{\vec{G}''}
\tilde{u}_{m\vec{k}}(\vec{G}'')
e^{i(\vec{k}+\vec{G}'')\cdot\vec{r}}
\right]
d\vec{r}
\\\nonumber
&=
\sum_{\vec{G}}
\sum_{\vec{G}'}
\sum_{\vec{G}''}
\tilde{u}_{n\vec{k}+\vec{q}}^*(\vec{G})
\left[
-i(\mathrm{q}_{\alpha}+\mathrm{G}'_{\alpha})
\tilde{v}_{s}(\vec{q}+\vec{G}')
\right]
\tilde{u}_{m\vec{k}}(\vec{G}'')
\\
&\phantom{=}
\times
\int
e^{-i(\vec{G}-\vec{G}'-\vec{G}'')\cdot\vec{r}}
d\vec{r}
\\
&=
\sum_{\vec{G}}
\sum_{\vec{G}'}
\sum_{\vec{G}''}
\tilde{u}_{n\vec{k}+\vec{q}}^*(\vec{G})
\left[
-i(\mathrm{q}_{\alpha}+\mathrm{G}'_{\alpha})
\tilde{v}_{s}(\vec{q}+\vec{G}')
\right]
\tilde{u}_{m\vec{k}}(\vec{G}'')
\delta_{\vec{G}-\vec{G}'',\vec{G}'}
\\
&=
\sum_{\vec{G}}
\sum_{\vec{G}''}
\tilde{u}_{n\vec{k}+\vec{q}}^*(\vec{G})
\left[
-i(\mathrm{q}_{\alpha}+\mathrm{G}_{\alpha}-\mathrm{G}''_{\alpha})
\tilde{v}_{s}(\vec{q}+\vec{G}-\vec{G}'')
\right]
\tilde{u}_{m\vec{k}}(\vec{G}'')
.\end{align}

Or another way of obtaining the same:
\begin{align}
\langle\psi_{n\vec{k}+\vec{q}}|
\partial_{s,\vec{q}}^{\alpha} V_{loc}
|\psi_{m\vec{k}}\rangle
&=
\sum_{\vec{G}}
\sum_{\vec{G}'}
\langle \psi_{n\vec{k}+\vec{q}} | \vec{k}+\vec{q}+\vec{G} \rangle
\langle \vec{k}+\vec{q}+\vec{G} | \partial_{s,\vec{q}}^{\alpha} V_{loc} | \vec{k}+\vec{G}' \rangle
\langle \vec{k}+\vec{G}' | \psi_{m\vec{k}} \rangle
\\\nonumber
&=
\sum_{\vec{G}}
\sum_{\vec{G}'}
\tilde{u}_{n\vec{k}+\vec{q}}^*(\vec{G})
\\\nonumber
&\hspace{1.5cm}
\times
\left[
\int\int\langle \vec{k}+\vec{q}+\vec{G} | \vec{r} \rangle \langle \vec{r} | \partial_{s,\vec{q}}^{\alpha} V_{loc} | \vec{r} \rangle \langle \vec{r}' | \vec{k}+\vec{G}' \rangle
d\vec{r}
d\vec{r}'
\right]
\\
&\hspace{1.5cm}
\times
\tilde{u}_{m\vec{k}}(\vec{G}')
\\\nonumber
&=
\sum_{\vec{G}}
\sum_{\vec{G}'}
\tilde{u}_{n\vec{k}+\vec{q}}^*(\vec{G})
\\\nonumber
&\hspace{1.5cm}
\times
\left[
\int\int
e^{-i(\vec{k}+\vec{q}+\vec{G})\cdot\vec{r}}
\partial_{s,\vec{q}}^{\alpha} V_{loc}(\vec{r})
\delta(\vec{r}-\vec{r}')
e^{i(\vec{k}+\vec{G}')\cdot\vec{r}'}
d\vec{r}
d\vec{r}'
\right]
\\
&\hspace{1.5cm}
\times
\tilde{u}_{m\vec{k}}(\vec{G}')
\\\nonumber
&=
\sum_{\vec{G}}
\sum_{\vec{G}'}
\tilde{u}_{n\vec{k}+\vec{q}}^*(\vec{G})
\\\nonumber
&\hspace{1.5cm}
\times
\left[
\int
e^{-i(\vec{q}+\vec{G}-\vec{G}')\cdot\vec{r}}
\partial_{s,\vec{q}}^{\alpha} V_{loc}(\vec{r})
d\vec{r}
\right]
\\
&\hspace{1.5cm}
\times
\tilde{u}_{m\vec{k}}(\vec{G}')
\\\label{eq:dvpsi-combolution-1}
&=
\sum_{\vec{G}}
\sum_{\vec{G}'}
\tilde{u}_{n\vec{k}+\vec{q}}^*(\vec{G})
\partial_{s,\vec{q}}^{\alpha} V_{loc}(\vec{q}+\vec{G}-\vec{G}')
\tilde{u}_{m\vec{k}}(\vec{G}')
\\\label{eq:dvpsi-combolution-2}
&=
\sum_{\vec{G}}
\sum_{\vec{G}'}
\tilde{u}_{n\vec{k}+\vec{q}}^*(\vec{G})
\left[
-i
(\mathrm{q}_{\alpha}+\mathrm{G}_{\alpha}-\mathrm{G}'_{\alpha})
\tilde{v}_{s}(\vec{q}+\vec{G}-\vec{G}')
\right]
\tilde{u}_{m\vec{k}}(\vec{G}')
.\end{align}

\subsubsection*{Implementation notes}

Within intw, the matrix element of the local part is computed in two steps:
\begin{enumerate}
\item The wave function $\psi_{m\vec{k}}$ and the local potential $\partial_{s,\vec{q}}^{\alpha} \tilde{V}_{loc}$ are multiplied in real space and stored in the variable $\verb|vpsi|$:
\begin{equation}
\verb|vpsi|(\vec{r})
=
\partial_{s,\vec{q}}^{\alpha} \tilde{V}_{loc}(\vec{r})\psi_{m\vec{k}}(\vec{r})
\end{equation}
and then $\verb|vpsi|(\vec{G})$ is obtained by makin the Fourier transform. This is equivalent of doing the combolution in Fourier space of eq. \eqref{eq:dvpsi-combolution-1} and  \eqref{eq:dvpsi-combolution-1}, which is
\begin{equation}
\verb|vpsi|(\vec{G})
=
\sum_{\vec{G}'}
\partial_{s,\vec{q}}^{\alpha} \tilde{V}_{loc}(\vec{q}+\vec{G}-\vec{G}')
\tilde{u}_{m\vec{k}}(\vec{G}')
.\end{equation}
\item And then, the matrix element is computed by projecting $\verb|vpsi|$ into $\psi_{n\vec{k}+\vec{q}}$ in Fourier space:
\begin{equation}
\langle\psi_{n\vec{k}+\vec{q}}|
\partial_{s,\vec{q}}^{\alpha} V_{loc}
|\psi_{m\vec{k}}\rangle
=
\sum_{\vec{G}}
\tilde{u}_{n\vec{k}+\vec{q}}^*(\vec{G})
\verb|vpsi|(\vec{G})
.\end{equation}
\end{enumerate}

\subsection*{Non-local part}

Following the same procedure with the non-local part of the pseudo-potential:
\begin{align}
V_{NL}^{\vec{q}}(\vec{r}'.\vec{r}'')
&=
\sum_{s,\vec{R}}
\sum_{i,i'}
D_{si,si'}
\beta_{si}(\vec{r}'-\boldsymbol{\tau}_{s}-\vec{R}-\vec{u}_{s,\vec{q}}
e^{i\vec{q}\cdot\vec{R}})
\beta_{si'}^*(\vec{r}''-\boldsymbol{\tau}_{s}-\vec{R}-\vec{u}_{s,\vec{q}}
e^{i\vec{q}\cdot\vec{R}})
\\
&=
\sum_{\vec{q}',\vec{q}''\in\text{1BZ}}
\sum_{\vec{G}',\vec{G}''}
\tilde{V}_{NL}^{\vec{q}}(\vec{q}'+\vec{G}',\vec{q}''+\vec{G}'')
e^{i(\vec{q}'+\vec{G}')\cdot\vec{r}'}
e^{-i(\vec{q}''+\vec{G}'')\cdot\vec{r}''}
,\end{align}
where $\vec{q}$ and $\vec{q}$ satisfy eq.~\eqref{eq:Born-von_Karman-q-vectors}, and
\begin{align}
\tilde{V}_{NL}^{\vec{q}}(\vec{q}'+\vec{G}',\vec{q}''+\vec{G}'')
&=
\frac{1}{\Omega_{SC}^2}
\int_{\Omega_{SC}}
\int_{\Omega_{SC}}
V_{NL}^{\vec{q}}(\vec{r}',\vec{r}'')
e^{-i(\vec{q}'+\vec{G}')\cdot\vec{r}'}
e^{i(\vec{q}''+\vec{G}'')\cdot\vec{r}''}
d\vec{r}'
d\vec{r}''
\\\nonumber
&=
\frac{1}{\Omega_{SC}^2}
\sum_{s,\vec{R}}
\sum_{i,i'}
D_{si,si}
\int_{\Omega_{SC}}
\beta_{si}(\vec{r}'-\boldsymbol{\tau}_{s}-\vec{R}-\vec{u}_{s,\vec{q}}
e^{i\vec{q}\cdot\vec{R}})
\\\nonumber
&\hspace{4.5cm}
\times
e^{-i(\vec{q}'+\vec{G}')\cdot\vec{r}'}
d\vec{r}'
\\\nonumber
&\hspace{3.2cm}
\times
\int_{\Omega_{SC}}
\beta^*_{si'}(\vec{r}''-\boldsymbol{\tau}_{s}-\vec{R}-\vec{u}_{s,\vec{q}}e^{i\vec{q}\cdot\vec{R}})
\\
&\hspace{4.5cm}
\times
e^{i(\vec{q}''+\vec{G}'')\cdot\vec{r}''}
d\vec{r}''
\\\nonumber
&=
\frac{1}{\Omega_{SC}^2}
\sum_{S\in{SC}}
\sum_{\vec{R}_{SC}}
\sum_{i,i'}
D_{Si,Si'}
\int_{\Omega_{SC}}
\beta_{Si}(\vec{r}'-\boldsymbol{\tau}_{S}-\vec{R}_{SC}-\vec{u}_{S,\vec{q}})
\\\nonumber
&\hspace{5.5cm}
\times
e^{-i(\vec{q}'+\vec{G}')\cdot\vec{r}'}
d\vec{r}'
\\\nonumber
&\hspace{4.3cm}
\times
\int_{\Omega_{SC}}
\beta^*_{Si'}(\vec{r}''-\boldsymbol{\tau}_{S}-\vec{R}_{SC}-\vec{u}_{S,\vec{q}})
\\
&\hspace{5.5cm}
\times
e^{i(\vec{q}''+\vec{G}'')\cdot\vec{r}''}
d\vec{r}''
\\\nonumber
&=
\frac{1}{\Omega_{SC}^2}
\sum_{S\in{SC}}
\sum_{\vec{R}_{SC}}
\sum_{i,i'}
D_{Si,Si'}
\int_{\Omega_{SC}(-\vec{R}_{SC})}
\beta_{Si}(\vec{r}'_1-\boldsymbol{\tau}_{S}-\vec{u}_{S,\vec{q}})
\\\nonumber
&\hspace{5.5cm}
\times
e^{-i(\vec{q}'+\vec{G}')\cdot\vec{r}'_1}
d\vec{r}'_1
\\\nonumber
&\hspace{4.3cm}
\times
\int_{\Omega_{SC}(-\vec{R}_{SC})}
\beta^*_{Si'}(\vec{r}'_2-\boldsymbol{\tau}_{S}-\vec{u}_{S,\vec{q}})
\\
&\hspace{5.5cm}
\times
e^{i(\vec{q}''+\vec{G}'')\cdot\vec{r}'_2}
d\vec{r}'_2
\\\nonumber
&=
\frac{1}{\Omega_{SC}^2}
\sum_{S\in{SC}}
\sum_{i,i'}
D_{Si,Si'}
\int_{\text{full space}}
\beta_{Si}(\vec{r}'_1-\boldsymbol{\tau}_{S}-\vec{u}_{S,\vec{q}})
\\\nonumber
&\hspace{5.5cm}
\times
e^{-i(\vec{q}'+\vec{G}')\cdot\vec{r}'_1}
d\vec{r}'_1
\\\nonumber
&\hspace{3.5cm}
\times
\int_{\text{full space}}
\beta^*_{Si'}(\vec{r}'_2-\boldsymbol{\tau}_{S}-\vec{u}_{S,\vec{q}})
\\
&\hspace{5.5cm}
\times
e^{i(\vec{q}''+\vec{G}'')\cdot\vec{r}'_2}
d\vec{r}'_2
\\\nonumber
&=
\frac{1}{\Omega_{SC}^2}
\sum_{S\in{SC}}
e^{-i(\vec{q}'+\vec{G}')\cdot(\boldsymbol{\tau}_{S}+\vec{u}_{S,\vec{q}})}
e^{i(\vec{q}''+\vec{G}'')\cdot(\boldsymbol{\tau}_{S}+\vec{u}_{S,\vec{q}})}
\\\nonumber
&\hspace{1.1cm}
\times
\sum_{i,i'}
D_{Si,Si'}
\int_{\text{full space}}
\beta_{Si}(\vec{r}''_1)
e^{-i(\vec{q}'+\vec{G}')\cdot\vec{r}''_1}
d\vec{r}''_1
\\
&\hspace{2.8cm}
\times
\int_{\text{full space}}
\beta^*_{Si'}(\vec{r}''_2)
e^{i(\vec{q}''+\vec{G}'')\cdot\vec{r}''_2}
d\vec{r}''_2
\\\nonumber
&=
\frac{1}{N_{SC}^2}
\sum_{S\in{SC}}
e^{-i(\vec{q}'+\vec{G}')\cdot\vec{u}_{S,\vec{q}}}
e^{i(\vec{q}''+\vec{G}'')\cdot\vec{u}_{S,\vec{q}}}
\\
&\hspace{1.1cm}
\times
\sum_{i,i'}
D_{Si,Si'}
\tilde{\beta}_{Si}(\vec{q}'+\vec{G}')
\tilde{\beta}^*_{Si'}(\vec{q}''+\vec{G}'')
\\\nonumber
&=
\frac{1}{N_{SC}^2}
\sum_{s\in{UC}}
\sum_{\vec{R}\in{SC}}
e^{-i(\vec{q}'+\vec{G}')\cdot\vec{u}_{s,\vec{q}}e^{i\vec{q}\cdot\vec{R}}}
e^{i(\vec{q}''+\vec{G}'')\cdot\vec{u}_{s,\vec{q}}e^{i\vec{q}\cdot\vec{R}}}
\\
&\hspace{1.1cm}
\times
\sum_{i,i'}
D_{si,si'}
e^{-i(\vec{q}'+\vec{G}')\cdot\vec{R}}
\tilde{\beta}_{si}(\vec{q}'+\vec{G}')
e^{i(\vec{q}''+\vec{G}'')\cdot\vec{R}}
\tilde{\beta}^*_{si'}(\vec{q}''+\vec{G}'')
\\\nonumber
&=
\frac{1}{N_{SC}^2}
\sum_{s\in{UC}}
\sum_{\vec{R}\in{SC}}
e^{-i(\vec{q}'-\vec{q}'')\cdot\vec{R}}
e^{-i(\vec{q}'+\vec{G}')\cdot\vec{u}_{s,\vec{q}}e^{i\vec{q}\cdot\vec{R}}}
e^{i(\vec{q}''+\vec{G}'')\cdot\vec{u}_{s,\vec{q}}e^{i\vec{q}\cdot\vec{R}}}
\\
&\hspace{1.1cm}
\times
\sum_{i,i'}
D_{si,si'}
\tilde{\beta}_{si}(\vec{q}'+\vec{G}')
\tilde{\beta}^*_{si'}(\vec{q}''+\vec{G}'')
.\end{align}

And the first derivative of $V_{NL}^{\vec{q}}$ with respect to the phonon polarization $\vec{u}_{s,\vec{q}}$ along the cartesian direction $\alpha$ around the atomic equilibrium positions:
\begin{align}
\partial_{s,\vec{q}}^{\alpha} V_{NL}(\vec{r}',\vec{r}'')
&\equiv
\frac{d V_{NL}^{\vec{q}}(\vec{r}',\vec{r}'')}{d \mathrm{u}_{s,\vec{q}}^{\alpha}}
|_{\vec{u}=0}
\\
&=
\sum_{\vec{q}',\vec{q}''\in\text{1BZ}}
\sum_{\vec{G}',\vec{G}''}
\frac{d \tilde{V}_{NL}^{\vec{q}}(\vec{q}'+\vec{G}',\vec{q}''+\vec{G}'')}{d \mathrm{u}_{s,\vec{q}}^{\alpha}}
|_{\vec{u}=0}
e^{i(\vec{q}'+\vec{G}')\cdot\vec{r}'}
e^{-i(\vec{q}''+\vec{G}'')\cdot\vec{r}''}
,\end{align}
where
\begin{align}
\partial_{s,\vec{q}}^{\alpha} V_{NL}(\vec{q}'+\vec{G}',\vec{q}''+\vec{G}'')
&\equiv
\frac{d \tilde{V}_{NL}^{\vec{q}}(\vec{q}'+\vec{G}',\vec{q}''+\vec{G}'')}{d \mathrm{u}_{s,\vec{q}}^{\alpha}}
|_{\vec{u}=0}
\\\nonumber
&=
\frac{1}{N_{SC}^2}
\sum_{s'\in{UC}}
\sum_{\vec{R}\in{SC}}
e^{-i(\vec{q}'-\vec{q}'')\cdot\vec{R}}
\\\nonumber
&\hspace{2.0cm}
\times
\frac{d }{d \mathrm{u}_{s,\vec{q}}^{\alpha}}
\left[
e^{-i(\vec{q}'+\vec{G}')\cdot\vec{u}_{s',\vec{q}}e^{i\vec{q}\cdot\vec{R}}}
e^{i(\vec{q}''+\vec{G}'')\cdot\vec{u}_{s',\vec{q}}e^{i\vec{q}\cdot\vec{R}}}
\right]
|_{\vec{u}=0}
\\
&\hspace{2.0cm}
\times
\sum_{i,i'}
D_{s'i,s'i'}
\tilde{\beta}_{s'i}(\vec{q}'+\vec{G}')
\tilde{\beta}^*_{s'i'}(\vec{q}''+\vec{G}'')
\\\nonumber
&=
\frac{1}{N_{SC}^2}
\sum_{\vec{R}\in{SC}}
e^{-i(\vec{q}'-\vec{q}'')\cdot\vec{R}}
\\\nonumber
&\hspace{2.0cm}
\times
\left[
-i(\mathrm{q}'_{\alpha}+\mathrm{G}'_{\alpha}-\mathrm{q}''_{\alpha}-\mathrm{G}''_{\alpha})
e^{i\vec{q}\cdot\vec{R}}
\right]
\\
&\hspace{2.0cm}
\times
\sum_{i,i'}
D_{si,si'}
\tilde{\beta}_{si}(\vec{q}'+\vec{G}')
\tilde{\beta}^*_{si'}(\vec{q}''+\vec{G}'')
\\\nonumber
&=
-i(\mathrm{q}'_{\alpha}+\mathrm{G}'_{\alpha}-\mathrm{q}''_{\alpha}-\mathrm{G}''_{\alpha})
\frac{1}{N_{SC}}
\sum_{i,i'}
D_{si,si'}
\\\nonumber
&\hspace{2.0cm}
\times
\tilde{\beta}_{si}(\vec{q}'+\vec{G}')
\tilde{\beta}^*_{si'}(\vec{q}''+\vec{G}'')
\\
&\hspace{2.0cm}
\times
\frac{1}{N_{SC}}
\sum_{\vec{R}\in{SC}}
e^{-i(\vec{q}'-\vec{q}''-\vec{q})\cdot\vec{R}}
\\\nonumber
&=
-i(\mathrm{q}'_{\alpha}+\mathrm{G}'_{\alpha}-\mathrm{q}''_{\alpha}-\mathrm{G}''_{\alpha})
\frac{1}{N_{SC}}
\sum_{i,i'}
D_{si,si'}
\\\nonumber
&\hspace{2.0cm}
\times
\tilde{\beta}_{si}(\vec{q}'+\vec{G}')
\tilde{\beta}^*_{si'}(\vec{q}''+\vec{G}'')
\\
&\hspace{2.0cm}
\times
\delta_{\vec{q}'-\vec{q}'',\vec{q}}
,\end{align}


Therefore:
\begin{align}
\nonumber
\partial_{s,\vec{q}}^{\alpha} V_{NL}(\vec{r}',\vec{r}'')
&=
\frac{1}{N_{SC}}
\sum_{\vec{q}',\vec{q}''\in\text{1BZ}}
\sum_{\vec{G}',\vec{G}''}
-i(\mathrm{q}'_{\alpha}+\mathrm{G}'_{\alpha}-\mathrm{q}''_{\alpha}-\mathrm{G}''_{\alpha})
\sum_{i,i'}
D_{si,si'}
\tilde{\beta}_{si}(\vec{q}'+\vec{G}')
\tilde{\beta}^*_{si'}(\vec{q}''+\vec{G}'')
\\
&\qquad\qquad\qquad
\times
\delta_{\vec{q}'-\vec{q}'',\vec{q}}
e^{i(\vec{q}'+\vec{G}')\cdot\vec{r}'}
e^{-i(\vec{q}''+\vec{G}'')\cdot\vec{r}''}
\\\nonumber
&=
\frac{1}{N_{SC}}
\sum_{\vec{q}''\in\text{1BZ}}
\sum_{\vec{G}',\vec{G}''}
-i(\mathrm{q}_{\alpha}+\mathrm{G}'_{\alpha}-\mathrm{G}''_{\alpha})
\sum_{i,i'}
D_{si,si'}
\tilde{\beta}_{si}(\vec{q}+\vec{q}''+\vec{G}')
\tilde{\beta}^*_{si'}(\vec{q}''+\vec{G}'')
\\
&\qquad\qquad\qquad
\times
e^{i(\vec{q}+\vec{q}''+\vec{G}')\cdot\vec{r}'}
e^{-i(\vec{q}''+\vec{G}'')\cdot\vec{r}''}
,\end{align}




Finally, the electron-phonon matrix element for the non-local part of the pseudo-potential is:
\begin{align}
\langle\psi_{n\vec{k}+\vec{q}}|
\partial_{s,\vec{q}}^{\alpha} V_{NL}
|\psi_{m\vec{k}}\rangle
&=
\int_{\Omega_{SC}}
\int_{\Omega_{SC}}
\langle \psi_{n\vec{k}+\vec{q}} | \vec{r}' \rangle
\langle \vec{r}' | \partial_{s,\vec{q}}^{\alpha} V_{NL} | \vec{r}'' \rangle
\langle \vec{r}'' | \psi_{m\vec{k}} \rangle
d\vec{r}'
d\vec{r}''
\\
&=
\int_{\Omega_{SC}}
\int_{\Omega_{SC}}
\psi_{n\vec{k}+\vec{q}}^*(\vec{r}')
\partial_{s,\vec{q}}^{\alpha} V_{NL}(\vec{r}',\vec{r}'')
\psi_{m\vec{k}}(\vec{r}'')
d\vec{r}'
d\vec{r}''
\\\nonumber
&=
\int_{\Omega_{SC}}
\int_{\Omega_{SC}}
\left[
\sum_{\vec{G}}
\tilde{u}_{n\vec{k}+\vec{q}}(\vec{G})
e^{i(\vec{k}+\vec{q}+\vec{G})\cdot\vec{r}'}
\right]^*
\\\nonumber
&\phantom{=\int\int}
\times
\left[
\frac{1}{N_{SC}}
\sum_{\vec{q}''\in\text{1BZ}}
\sum_{\vec{G}',\vec{G}''}
-i(\mathrm{q}_{\alpha}+\mathrm{G}'_{\alpha}-\mathrm{G}''_{\alpha})
\right.
\\\nonumber
&\phantom{=\int\int\Big[}
\times
\sum_{i,i'}
D_{si,si'}
\tilde{\beta}_{si}(\vec{q}+\vec{q}''+\vec{G}')
\tilde{\beta}^*_{si'}(\vec{q}''+\vec{G}'')
\\\nonumber
&\phantom{=\int\int}
\times
\left.
e^{i(\vec{q}+\vec{q}''+\vec{G}')\cdot\vec{r}'}
e^{-i(\vec{q}''+\vec{G}'')\cdot\vec{r}''}
\right]
\\
&
\phantom{=\int\int}
\times
\left[
\sum_{\vec{G}'''}
\tilde{u}_{m\vec{k}}(\vec{G}''')
e^{i(\vec{k}+\vec{G}''')\cdot\vec{r}''}
\right]
d\vec{r}'
d\vec{r}''
\\\nonumber
&=
\frac{1}{N_{SC}}
\sum_{\vec{q}'\in\text{1BZ}}
\sum_{\vec{G},\vec{G}',\vec{G}'',\vec{G}'''}
-i(\mathrm{q}_{\alpha}+\mathrm{G}'_{\alpha}-\mathrm{G}''_{\alpha})
\\\nonumber
&\qquad
\times
\sum_{i,i'}
D_{si,si'}
\tilde{u}^*_{n\vec{k}+\vec{q}}(\vec{G})
\tilde{\beta}_{si}(\vec{q}+\vec{q}''+\vec{G}')
\tilde{\beta}^*_{si'}(\vec{q}''+\vec{G}'')
\tilde{u}_{m\vec{k}}(\vec{G}''')
\\\nonumber
&\qquad
\times
\int_{\Omega_{SC}}
e^{-i(\vec{k}+\vec{G})\cdot\vec{r}'}
e^{i(\vec{q}'+\vec{G}')\cdot\vec{r}'}
d\vec{r}'
\\
&\qquad
\times
\int_{\Omega_{SC}}
e^{-i(\vec{q}'+\vec{G}'')\cdot\vec{r}''}
e^{i(\vec{k}+\vec{G}''')\cdot\vec{r}''}
d\vec{r}''
\\\nonumber
&=
\frac{1}{N_{SC}}
\sum_{\vec{q}'\in\text{1BZ}}
\sum_{\vec{G},\vec{G}',\vec{G}'',\vec{G}'''}
-i(\mathrm{q}_{\alpha}+\mathrm{G}'_{\alpha}-\mathrm{G}''_{\alpha})
\\\nonumber
&\qquad
\times
\sum_{i,i'}
D_{si,si'}
\tilde{u}^*_{n\vec{k}+\vec{q}}(\vec{G})
\tilde{\beta}_{si}(\vec{q}+\vec{q}''+\vec{G}')
\tilde{\beta}^*_{si'}(\vec{q}''+\vec{G}'')
\tilde{u}_{m\vec{k}}(\vec{G}''')
\\\nonumber
&\qquad
\times
\int_{\Omega_{SC}}
e^{-i(\vec{k}+\vec{G})\cdot\vec{r}'}
e^{i(\vec{q}'+\vec{G}')\cdot\vec{r}'}
d\vec{r}'
\\
&\qquad
\times
\delta_{\vec{k}-\vec{q}'+\vec{G}''',\vec{G}''}
\\\nonumber
&=
\frac{1}{N_{SC}}
\sum_{\vec{q}'\in\text{1BZ}}
\sum_{\vec{G},\vec{G}',\vec{G}'''}
-i(\mathrm{q}_{\alpha}+\mathrm{G}'_{\alpha}-\mathrm{k}_{\alpha}+\mathrm{q}'_{\alpha}-\mathrm{G}'''_{\alpha})
\\\nonumber
&\qquad
\times
\sum_{i,i'}
D_{si,si'}
\tilde{u}^*_{n\vec{k}+\vec{q}}(\vec{G})
\tilde{\beta}_{si}(\vec{q}+\vec{q}'+\vec{G}')
\tilde{\beta}^*_{si'}(\vec{q}'+\vec{k}-\vec{q}'+\vec{G}''')
\tilde{u}_{m\vec{k}}(\vec{G}''')
\\
&\qquad
\times
\int_{\Omega_{SC}}
e^{-i(\vec{k}+\vec{G})\cdot\vec{r}'}
e^{i(\vec{q}'+\vec{G}')\cdot\vec{r}'}
d\vec{r}'
\\\nonumber
&=
\frac{1}{N_{SC}}
\sum_{\vec{q}'\in\text{1BZ}}
\sum_{\vec{G},\vec{G}',\vec{G}'''}
-i(\mathrm{q}_{\alpha}+\mathrm{G}'_{\alpha}-\mathrm{k}_{\alpha}+\mathrm{q}'_{\alpha}-\mathrm{G}'''_{\alpha})
\\\nonumber
&\qquad
\times
\sum_{i,i'}
D_{si,si'}
\tilde{u}^*_{n\vec{k}+\vec{q}}(\vec{G})
\tilde{\beta}_{si}(\vec{q}+\vec{q}'+\vec{G}')
\tilde{\beta}^*_{si'}(\vec{q}'+\vec{k}-\vec{q}'+\vec{G}''')
\tilde{u}_{m\vec{k}}(\vec{G}''')
\\
&\qquad
\times
\delta_{\vec{k}+\vec{G}-\vec{q}',\vec{G}'}
\\\nonumber
&=
\frac{1}{N_{SC}}
\sum_{\vec{q}'\in\text{1BZ}}
\sum_{\vec{G},\vec{G}'''}
-i(\mathrm{q}_{\alpha}+\mathrm{G}_{\alpha}-\mathrm{G}'''_{\alpha})
\\
&\qquad
\times
\sum_{i,i'}
D_{si,si'}
\tilde{u}^*_{n\vec{k}+\vec{q}}(\vec{G})
\tilde{\beta}_{si}(\vec{k}+\vec{q}+\vec{G})
\tilde{\beta}^*_{si'}(\vec{k}+\vec{G}''')
\tilde{u}_{m\vec{k}}(\vec{G}''')
\\\nonumber
&=
\sum_{\vec{G},\vec{G}'''}
-i(\mathrm{q}_{\alpha}+\mathrm{G}_{\alpha}-\mathrm{G}'''_{\alpha})
\\
&\qquad
\times
\sum_{i,i'}
D_{si,si'}
\tilde{u}^*_{n\vec{k}+\vec{q}}(\vec{G})
\tilde{\beta}_{si}(\vec{k}+\vec{q}+\vec{G})
\tilde{\beta}^*_{si'}(\vec{k}+\vec{G}''')
\tilde{u}_{m\vec{k}}(\vec{G}''')
\\\nonumber
&=
\sum_{i,i'}
D_{si,si'}
\sum_{\vec{G}}
\tilde{u}^*_{n\vec{k}+\vec{q}}(\vec{G})
\\\nonumber
&\quad
\times
\sum_{\vec{G}'''}
\left[
\phantom{-+\mathrm{q}_{\alpha}}
i(\mathrm{k}_{\alpha}+\mathrm{G}'''_{\alpha})
\tilde{\beta}_{s,i}(\vec{k}+\vec{q}+\vec{G})
\tilde{\beta}^*_{s,i'}(\vec{k}+\vec{G}''')
\tilde{u}_{m\vec{k}}(\vec{G}''')
\right.
\\
&\quad
\phantom{\times''\sum_{\vec{G}'''}\big[}
\left.
-i(\mathrm{k}_{\alpha}+\mathrm{q}_{\alpha}+\mathrm{G}_{\alpha})
\tilde{\beta}_{s,i}(\vec{k}+\vec{q}+\vec{G})
\tilde{\beta}^*_{s,i'}(\vec{k}+\vec{G}''')
\tilde{u}_{m\vec{k}}(\vec{G}''')
\right]
\\\nonumber
&=
\sum_{i,i'}
D_{si,si'}
\sum_{\vec{G}}
\tilde{u}^*_{n\vec{k}+\vec{q}}(\vec{G})
\\\nonumber
&\quad
\times
\sum_{\vec{G}'''}
\left[
\phantom{-+\mathrm{q}_{\alpha}}
\tilde{\beta}_{s,i}(\vec{k}+\vec{q}+\vec{G})
\left\{
i(\mathrm{k}_{\alpha}+\mathrm{G}'''_{\alpha})
\tilde{\beta}^*_{s,i'}(\vec{k}+\vec{G}''')
\right\}
\tilde{u}_{m\vec{k}}(\vec{G}''')
\right.
\\
&\quad
\phantom{\times''\sum_{\vec{G}'''}\big[}
\left.
-
\left\{
i(\mathrm{k}_{\alpha}+\mathrm{q}_{\alpha}+\mathrm{G}_{\alpha})
\tilde{\beta}_{s,i}(\vec{k}+\vec{q}+\vec{G})
\right\}
\tilde{\beta}^*_{s,i'}(\vec{k}+\vec{G}''')
\tilde{u}_{m\vec{k}}(\vec{G}''')
\right]
\end{align}



\subsubsection*{Implementation notes}

Within intw, the matrix element of the non-local part is also computed in several steps:
\begin{enumerate}
\item The variables $\verb|projec_1|(i')$ and $\verb|projec_2|(i')$ are computed by projecting the wave function $\psi_{m\vec{k}}$ is projected into ${\beta}_{s,i}^*$ and it's derivative:
\begin{equation}
\verb|projec_1|(i')
=
\sum_{\vec{G}'''}
\tilde{\beta}_{s,i'}^*(\vec{k}+\vec{G}''')
\tilde{u}_{m\vec{k}}(\vec{G}''')
\end{equation}
\begin{equation}
\verb|projec_2|(i')
=
\sum_{\vec{G}'''}
i(\mathrm{k}_{\alpha}+\mathrm{G}'''_{\alpha})
\tilde{\beta}_{s,i'}^*(\vec{k}+\vec{G}''')
\tilde{u}_{m\vec{k}}(\vec{G}''')
.\end{equation}
\item Then, the variables $\verb|projec_1|(i')$ and $\verb|projec_2|(i')$ are multiplied by the $D_{si,si'}$ matrix to obtain $\verb|Dij_projec_1|(i)$ and $\verb|Dij_projec_2|(i)$:
\begin{equation}
\verb|Dij_projec_1|(i)
=
\sum_{i'}
D_{si,si'}
\verb|projec_1|(i')
,\end{equation}
\begin{equation}
\verb|Dij_projec_2|(i)
=
\sum_{i'}
D_{si,si'}
\verb|projec_2|(i')
,\end{equation}
\item Next, the variable $\verb|dvnl_psi|$ is computed by multiplying ${\beta}_{s,i}$ by $\verb|Dij_projec_2|(i)$ and its derivative by $\verb|Dij_projec_1|(i)$:
\begin{equation}
\begin{array}{r@{}l}
\verb|dvnl_psi|(\vec{G})
=
\sum_{i}
\Big[
&
\phantom{-i(\mathrm{k}_{\alpha}+\mathrm{q}_{\alpha}+\mathrm{G}_{\alpha})}
\tilde{\beta}_{s,i}(\vec{k}+\vec{q}+\vec{G})\times \verb|Dij_projec_2|(i)
\\
&
-
i(\mathrm{k}_{\alpha}+\mathrm{q}_{\alpha}+\mathrm{G}_{\alpha})
\tilde{\beta}_{s,i}(\vec{k}+\vec{q}+\vec{G})\times \verb|Dij_projec_1|(i)
\Big]
\end{array}
.\end{equation}
\item And finally, the matrix element is computed by projecting $dvnl_psi$ into $\psi_{n\vec{k}+\vec{q}}$ in Fourier space:
\begin{equation}
\langle\psi_{n\vec{k}+\vec{q}}|
\partial_{s,\vec{q}}^{\alpha} V_{NL}
|\psi_{m\vec{k}}\rangle
=
\sum_{\vec{G}}
\tilde{u}^*_{n\vec{k}+\vec{q}}(\vec{G})
\verb|dvnl_psi|(\vec{G})
.\end{equation}
\end{enumerate}



\end{document}
