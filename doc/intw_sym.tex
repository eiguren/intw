
\documentclass[aps,prb,preprint,superscriptaddress,showpacs,showkeys]{revtex4-2}  
%\documentclass[aps,prb,twocolumn,superscriptaddress,showpacs,showkeys]{revtex4-2}  
\bibliographystyle{apsrev4-2}

\usepackage[pdftex]{graphicx}
\usepackage{color}
\usepackage{amsmath}
\usepackage{multirow}
\usepackage{placeins}
\usepackage{braket}

\usepackage{changes}
%\usepackage[final]{changes}

\usepackage{xr}
%\externaldocument[SM-]{sm_dmi_v2}

\begin{document}

% Use the \preprint command to place your local institutional report
% number in the upper righthand corner of the title page in preprint mode.
% Multiple \preprint commands are allowed.
% Use the 'preprintnumbers' class option to override journal defaults
% to display numbers if necessary
%\preprint{}

%Title of paper
\title{Summary of symmetries in INTW}



% repeat the \author .. \affiliation etc. as needed
% \email, \thanks, \homepage, \altaffiliation all apply to the current
% author. Explanatory text should go in the []'s, actual e-mail
% address or url should go in the {}'s for \email and \homepage.
% Please use the appropriate macro foreach each type of information

% \affiliation command applies to all authors since the last
% \affiliation command. The \affiliation command should follow the
% other information
% \affiliation can be followed by \email, \homepage, \thanks as well.

\newcommand{\QUIMI}[0]{{
Departamento de Pol\'{\i}meros y Materiales Avanzados: 
F\'{\i}sica, Qu\'{\i}mica y Tecnolog\'{\i}a, Facultad de Qu\'{\i}mica UPV/EHU,
Apartado 1072, 20080 Donostia-San Sebasti\'an, Spain}}

\newcommand{\DIPC}[0]{{
Donostia International Physics Center, 
Paseo Manuel de Lardiz\'abal 4, 20018 Donostia-San Sebasti\'an, Spain}}

\newcommand{\CFM}[0]{{
Centro de F\'{\i}sica de Materiales CFM/MPC (CSIC-UPV/EHU), 
Paseo Manuel de Lardiz\'abal 5, 20018 Donostia-San Sebasti\'an, Spain}}

%\author{M. Blanco-Rey}
%\affiliation{\QUIMI}
%\affiliation{\DIPC}
%\affiliation{\CFM}


\date{\today}

%\begin{abstract}
%bla bla
%\end{abstract}

\maketitle

%======================================================


\section{Notation}

We consider $\mathbf{k} \in \mathcal{M}_f$ to be the wavevectors in the full mesh of the 1BZ and 
$\mathbf{k}_s \in \mathcal{M}_i$ to be the wavevectors in the irreducible wedge of the 1BZ.

The symmetry elements are $\mathcal{S} = \{R | \mathbf{\tau} \}$ without time reversal (TR)
or $\mathcal{S} = \mathcal{T} \{R| \mathbf{\tau} \}$ with 
time reversal, where $\{R|\tau \}$ is an element of the space group of the crystal such that it 
transforms positions as $\mathbf{r}' = R \mathbf{r} - \mathbf{\tau} $. 

The $\mathbf{k} \in \mathcal{M}_f$ are linked to the $\mathbf{k}_s \in \mathcal{M}_i$ vectors 
by symmetry operations. We say they are ``symlinked'' by an operation $\mathcal{S}$ when 
we can write $R \mathbf{k_s} = \mathbf{k} + \mathbf{G}_s$ with TR or 
$-R \mathbf{k_s} = \mathbf{k} + \mathbf{G}_s$ without TR, 
where $\mathbf{G}_s$ is a vector of the crystal reciprocal lattice. 

The choice of an $\mathcal{S}$ operation to connect a $k$ and a $k_s$ may not be unique. 
In the INTW implementation we nevertheless set a given symlink, which will fix the 
wavefunction phase.

Transformation by $\mathcal{S}=\{ R|\tau \}$ of a real-space vector, a wavefunction and 
an operator:
\begin{align}
\mathbf{r}' & = \mathcal{S}^{-1} \mathbf{r} \mathcal{S} \equiv \mathcal{S}\mathbf{r} 
= \{ R|\tau \}  \mathbf{r} = R  \mathbf{r} - \tau \\
  \psi'( \mathbf{r} ) & = \psi ( \mathcal{S}^{-1} \mathbf{r} ) = \psi ( \{ R^{-1}| -R^{-1}\tau \}  \mathbf{r} ) = 
  \psi (R^{-1}(\mathbf{r}+\tau))  \\
\mathcal{O}'( \mathbf{r} ) &= \mathcal{S}^{-1} \mathcal{O}( \mathbf{r} ) \mathcal{S} = 
\mathcal{O}( \mathcal{S}^{-1} \mathbf{r} \mathcal{S} ) \equiv 
\mathcal{O}( \mathcal{S}\mathbf{r}) = \mathcal{O}(R  \mathbf{r} - \tau)
\end{align}

%======================================================

\section{Rotation of wavefunctions}

By Bloch's theorem, we write the wavefunctions as 
\begin{align}
\psi_{n\mathbf{k}} & = e^{i\mathbf{k}\cdot\mathbf{r}} u_{n\mathbf{k}} (\mathbf{r}) \\
u_{n\mathbf{k}} (\mathbf{r}) & = \sum_{\mathbf{G}} e^{i\mathbf{G}\cdot\mathbf{r}} 
     \tilde u_{n\mathbf{k}} (\mathbf{G})
\end{align}
where the sum runs over reciprocal lattice vectors.

Without TR, we have $R \mathbf{k_s} = \mathbf{k} + \mathbf{G}_s$:
\begin{align}
\{R| \mathbf{\tau} \} \psi_{n \mathbf{k_s}}(\mathbf{r}) & = \psi_{n \mathbf{k_s}}(R^{-1}(\mathbf{r}+\mathbf{\tau}))  = \\
& = e^{i\mathbf{k}_s \cdot R^{-1}(\mathbf{r}+\mathbf{\tau})} u_{n\mathbf{k}_s} (R^{-1}(\mathbf{r}+\mathbf{\tau})) = \\
& = e^{i\mathbf{k}_s \cdot R^{-1}(\mathbf{r}+\mathbf{\tau})} \sum_{\mathbf{G}}  e^{i\mathbf{G}\cdot R^{-1}(\mathbf{r}+\mathbf{\tau})} 
   \tilde u_{n\mathbf{k}_s} (\mathbf{G}) = \\
& = e^{i R \mathbf{k}_s \cdot (\mathbf{r}+\mathbf{\tau})} \sum_{\mathbf{G}}  e^{i R \mathbf{G}\cdot (\mathbf{r}+\mathbf{\tau}) }
   \tilde u_{n\mathbf{k}_s} (\mathbf{G}) = \\
& = e^{i \mathbf{k} \cdot (\mathbf{r}+\mathbf{\tau})} \sum_{\mathbf{G}} 
    e^{i ( R \mathbf{G} + {\mathbf{G}_s} ) \cdot (\mathbf{r}+\mathbf{\tau})} \tilde u_{n\mathbf{k}_s} (\mathbf{G}) = \\
& = e^{i \mathbf{k} \cdot \mathbf{r} } u_{n\mathbf{k}} (\mathbf{r}) = \psi_{n\mathbf{k}} (\mathbf{r}) 
\label{eq:rot_noTR1}
\end{align}
where
\begin{align}
u_{n\mathbf{k}} (\mathbf{r}) & =  \sum_{\mathbf{G}} e^{i\mathbf{G}\cdot\mathbf{r}} \tilde u_{n\mathbf{k}} (\mathbf{G}) \\
\tilde u_{n\mathbf{k}} (\mathbf{G}) & = e^{i(k + \mathbf{G}) \cdot \mathbf{\tau}} 
  \tilde u_{n\mathbf{k}_s} ( R^{-1}( \mathbf{G} - \mathbf{G}_s) )
\label{eq:rot_noTR2}
\end{align}


With TR, we have $-R \mathbf{k_s} = \mathbf{k} + \mathbf{G}_s$:
\begin{align}
\mathcal{T} \{R| \mathbf{\tau} \} \psi_{n \mathbf{k_s}}(\mathbf{r}) & =  \Big[ \{R| \mathbf{\tau} \} \psi_{n \mathbf{k_s}}(\mathbf{r}) \Big]^* = \\
& = \Big[ e^{i R \mathbf{k}_s \cdot (\mathbf{r}+\mathbf{\tau})} \sum_{\mathbf{G}}  
    e^{i R \mathbf{G}\cdot (\mathbf{r}+\mathbf{\tau}) } \tilde u_{n\mathbf{k}_s} (\mathbf{G})  \Big]^* = \\
& = e^{ (\mathbf{k} + \mathbf{G}_s) \cdot (\mathbf{r}+\mathbf{\tau}) } \sum_{\mathbf{G}}  
    e^{ - i R \mathbf{G}\cdot (\mathbf{r}+\mathbf{\tau}) } \tilde u_{n\mathbf{k}_s}^* (\mathbf{G}) = \\
& = e^{i \mathbf{k} \cdot \mathbf{r} } u_{n\mathbf{k}} (\mathbf{r}) = \psi_{n\mathbf{k}} (\mathbf{r}) 
\end{align}
where
\begin{align}
u_{n\mathbf{k}} (\mathbf{r}) & =  \sum_{\mathbf{G}} e^{i\mathbf{G}\cdot\mathbf{r}} \tilde u_{n\mathbf{k}} (\mathbf{G}) \\
\tilde u_{n\mathbf{k}} (\mathbf{G}) & = e^{i(k + \mathbf{G}) \cdot \mathbf{\tau}}  
  \tilde u_{n\mathbf{k}_s}^* ( R^{-1}( \mathbf{G}_s - \mathbf{G}) )
\end{align}

\textcolor{red}{rotate\_wfc\_test is implemented with $-\tau$, I think, but none of the tests 
seems to pick up on the difference. With my version of rotate\_wfc I get wrong $M_{mn}$ for the 
Si test, but I suspect that the error is a different one...}


As Lax states on page 293, the TR and point (or space) group operations commute. 
Indeed, in the equation above, we see that 
\begin{align}
\mathcal{T} \{R| \mathbf{\tau} \} \psi_{n \mathbf{k_s}}(\mathbf{r}) =  
    \Big[ \{R| \mathbf{\tau} \} \psi_{n \mathbf{k_s}}(\mathbf{r}) \Big]^* = 
\{R| \mathbf{\tau} \} \psi^*_{n \mathbf{k_s}}(\mathbf{r}) = 
    \{R| \mathbf{\tau} \} \mathcal{T} \psi_{n \mathbf{k_s}}(\mathbf{r})
\end{align}

\subsection{Spinor rotation}

This commutation rule is true also in 
the case of spinors. We write the spinor wavefunction as 
\begin{align}
\psi_{n\mathbf{k}}(\mathbf{r}) = \psi^\uparrow_{n\mathbf{k}}(\mathbf{r}) \ket{\uparrow} + 
  \psi^\downarrow_{n\mathbf{k}}(\mathbf{r}) \ket{\downarrow} \equiv 
\begin{pmatrix}
\psi^\uparrow_{n\mathbf{k}}(\mathbf{r}) \\
\psi^\downarrow_{n\mathbf{k}}(\mathbf{r}) 
\end{pmatrix}
\end{align}
Here, the transformations apply independently on the $\mathbf{r}$-dependent coordinates 
and spin-$\frac{1}{2}$ states. On the latter, the improper part and the fractional translation of
$\{R| \mathbf{\tau} \}$ have no effect. If we call $R_0$ to the pure rotation component of that 
transformation, consisting of a counterclockwise rotation of an $\alpha$ angle around direction 
$\hat n = (n_x, n_y, n_z)$, the rotated spin states will be given by the $2 \times 2$ matrix
\begin{align}
D_{1/2}(R_0) & = \cos\frac{\alpha}{2} \sigma_0 - i \sin\frac{\alpha}{2} \mathbf{\sigma}\cdot\hat{n} = \\
& = \begin{pmatrix}
\cos\frac{\alpha}{2} -i n_z \sin\frac{\alpha}{2}   &   -i\sin\frac{\alpha}{2} (n_z -i n_y ) \\
-i\sin\frac{\alpha}{2} (n_z +i n_y )  &  \cos\frac{\alpha}{2} +i n_z \sin\frac{\alpha}{2} \\
\end{pmatrix}
\end{align}
applied on $\ket{\uparrow}=\binom{1}{0}$ and $\ket{\downarrow}=\binom{0}{1}$, where
$\sigma_0$ is the $2 \times 2$ unit matrix and $\sigma_{x,y,z}$ are the Pauli matrices.

\textcolor{red}{This transformation of the spinor is like multiplying this matrix on the 
vector column of spinor components, not the proper part of the inverse $\{R| \mathbf{\tau} \}^{-1}$, 
which is implemented in INTW now and seems to work...}

With both point or space group element and TR, the spinor transforms as 
\begin{align}
\mathcal{T} \{R| \mathbf{\tau} \} \psi_{n\mathbf{k}}(\mathbf{r}) & = 
\{R| \mathbf{\tau} \} \mathcal{T} \psi_{n\mathbf{k}}(\mathbf{r}) =  \\
&= D_{1/2}(R_0) (i\sigma_y) 
   \Big( [ \psi^\uparrow_{n\mathbf{k}}(R^{-1}(\mathbf{r}+\mathbf{\tau})) ]^*  \ket{\uparrow} +
         [ \psi^\downarrow_{n\mathbf{k}}(R^{-1}(\mathbf{r}+\mathbf{\tau})) ]^* \ket{\downarrow}  \Big) = \\
&= [ \psi^\uparrow_{n\mathbf{k}}(R^{-1}(\mathbf{r}+\mathbf{\tau})) ]^*  
   D_{1/2}(R_0) (i\sigma_y) \ket{\uparrow} +
   [ \psi^\downarrow_{n\mathbf{k}}(R^{-1}(\mathbf{r}+\mathbf{\tau})) ]^* 
   D_{1/2}(R_0) (i\sigma_y) \ket{\downarrow} = \\
&= - [ \psi^\uparrow_{n\mathbf{k}}(R^{-1}(\mathbf{r}+\mathbf{\tau})) ]^*  
   D_{1/2}(R_0) \ket{\downarrow} +
   [ \psi^\downarrow_{n\mathbf{k}}(R^{-1}(\mathbf{r}+\mathbf{\tau})) ]^* 
   D_{1/2}(R_0) \ket{\uparrow} 
\end{align}
where $i\sigma_y$ is the TR operator for the spin degree of freedom.



%======================================================

\section{Rotation of matrix elements}

The electron-phonon (e-ph) matrix elements are defined as
\begin{align}
M_{\alpha s \mathbf{q}}^{nm}(\mathbf{k}) = \bra{\psi_{n \mathbf{k}+\mathbf{q}}} 
  \partial^\alpha_{s \mathbf{q}} V(\mathbf{r},\mathbf{r'}) \ket{\psi_{m \mathbf{k}}}
\end{align}
where $\mathbf{q}$ is the phonon wavevector $\alpha=x,y,z$ and $s$ is the displaced atom.
We want to generate by rotation the elements for $\mathbf{q}$ and $\mathbf{k}$ 
in the full 1BZs (lists $\mathcal{M}_f^q$ and $\mathcal{M}_f^k$, respectively), from the elements 
calculated on the irreducible phonon wavector list $\mathbf{q}_s \in \mathcal{M}_i^q$. 

Our criterion for phases is that the rotated e-ph matrix elements have to be equal to 
those calculated for brakets with wavefunctions $\psi_{n \mathbf{k}}$ and 
$\psi_{n \mathbf{k}+\mathbf{q}}$ obtained by ``symlink'' rotation of $k$-vectors in 
the irreducible list $\mathcal{M}_i^k$. 

For simplicity, we start by taking a $\mathcal{S}$ with no TR and consider only the local 
part of the matrix elements in the analysis.

Take a $\mathbf{q} \in \mathcal{M}_f^q$ such that it is ``symlinked'' to 
$\mathbf{q}_s \in \mathcal{M}_i^q$ by $\mathcal{S}= \{ R|\tau \}$. Therefore, 
$R \mathbf{q}_s = \mathbf{q} + \mathbf{G}_q$, with $\mathbf{G}_q$ a reciprocal lattice vector.
The inverse symmetry element is $\mathcal{S}^{-1} = \{ R^{-1} | -R^{-1}\tau \}$.
The local contribution matrix elements are
\begin{align}
M_{\alpha s \mathbf{q}}^{nm}(\mathbf{k}) & = \bra{ \mathcal{SS}^{-1} \psi_{n \mathbf{k}+\mathbf{q}}}   
  \partial^\alpha_{s \mathbf{q}} V^{loc} \ket{\mathcal{SS}^{-1} \psi_{m \mathbf{k}}} = \\
 &= \bra{ \mathcal{S}^{-1} \psi_{n \mathbf{k}+\mathbf{q}} } 
   \mathcal{S}^{-1} \partial^\alpha_{s \mathbf{q}} V^{loc} \mathcal{S}
   \ket{ \mathcal{S}^{-1} \psi_{m \mathbf{k}} } 
\end{align}
where we have used $\mathcal{S}^\dagger = \mathcal{S}^{-1}$ and (see Haritz's Eq. 78):
\begin{align}
\partial^\alpha_{s \mathbf{q}} V^{loc}(\mathbf{r}) = -i \sum_{\mathbf{G}} (q_\alpha + G_\alpha) 
  \tilde v_s (\mathbf{q}+\mathbf{G}) e^{ i(\mathbf{q}+\mathbf{G}) \cdot \mathbf{r} }
\end{align} 
Here we see that we need to check the effect of $\mathcal{S}$ on the wavefunctions and 
on the potential derivative.

\subsection{Effect of wavefunction rotation}

We consider the transformed wavefunctions by $\mathcal{S}^{-1}$. 
These are calculated as in Eqs.~\ref{eq:rot_noTR1} and \ref{eq:rot_noTR2}, but they do not constitute a symlink.
In the following, we use the notation $\psi,u$ to denote transformations that correspond to a symlink and $\phi,v$ 
for those that do not. 
Explicitly:
\begin{align}
R^{-1} \mathbf{k}  & = \mathbf{k}_1 + \mathbf{G}_1 \\
R^{-1}(\mathbf{k}+\mathbf{q}) & = R^{-1}\mathbf{k}+R^{-1}\mathbf{q}  = 
  \mathbf{k}_1 + \mathbf{q}_s + \mathbf{G}_1 - R^{-1}\mathbf{G}_q 
\end{align}
where $\mathbf{k}_1 \in \mathcal{M}_f^k$ and
\begin{align}
\phi_{n \mathbf{k}_1} (\mathbf{r}) & \equiv \mathcal{S}^{-1} \psi_{n \mathbf{k}} (\mathbf{r}) =  \\
&= e^{i \mathbf{k}_1 \cdot \mathbf{r} }
   \sum_\mathbf{G} e^{i \mathbf{G} \cdot \mathbf{r} } \tilde v_{n \mathbf{k}_1} (\mathbf{G}) \\
\tilde v_{n \mathbf{k}_1}+(\mathbf{G}) & =
   e^{-i(\mathbf{k}_1 \mathbf{G}) \cdot R^{-1}\tau }
   \tilde u_{n \mathbf{k}} ( R(\mathbf{G}-\mathbf{G}_1))
\end{align}
and
\begin{align}
\phi_{n \mathbf{k}_1 + \mathbf{q}_s} (\mathbf{r}) & \equiv 
   \mathcal{S}^{-1} \psi_{n \mathbf{k}+\mathbf{q}} (\mathbf{r}) = \\ 
&= e^{i (\mathbf{k}_1 + \mathbf{q}_s) \cdot \mathbf{r} }
   \sum_\mathbf{G} e^{i \mathbf{G} \cdot \mathbf{r} } \tilde v_{n \mathbf{k}_1 + \mathbf{q}_s} (\mathbf{G}) \\
\tilde v_{n \mathbf{k}_1 + \mathbf{q}_s} (\mathbf{G}) & = 
   e^{-i(\mathbf{k}_1 + \mathbf{q}_s +  \mathbf{G}) \cdot R^{-1}\tau }
   \tilde u_{n \mathbf{k} + \mathbf{q}} ( R(\mathbf{G}-\mathbf{G}_1+R^{-1}\mathbf{G}_q))  
\end{align}
The Bloch wavefunctions corresponding to states with wavevectors $\mathbf{k}_1$ and $\mathbf{k}_1+\mathbf{q}_s$
can be calculated via symlink to states with $\mathbf{k}_1^s$ and $\mathbf{k}_2^s$, respectively, 
through transformations $\mathcal{S}_1$ and $\mathcal{S}_2$, respectively, i.e.,
\begin{align}
\mathcal{S}_1 \psi_{n_1 \mathbf{k}_1^s} (\mathbf{r}) & = \psi_{n_1 \mathbf{k}_1} (\mathbf{r}) \\ 
\mathcal{S}_2 \psi_{n_2 \mathbf{k}_2^s} (\mathbf{r}) & = \psi_{n_2 \mathbf{k}_1 + \mathbf{q}_s} (\mathbf{r}) 
\end{align}
such that there is a unitary transformation relation between these and 
$\phi_{n \mathbf{k}_1} (\mathbf{r})$ and $\phi_{n \mathbf{k}_1 + \mathbf{q}_s} (\mathbf{r})$, respectively.
Therefore, we can write
\begin{align}
\ket{\phi_{n  \mathbf{k}_1}} & = \sum_{n_1}  
   \braket{ \psi_{n_1 \mathbf{k}_1} | \phi_{n  \mathbf{k}_1} } 
   \ket{ \psi_{n_1 \mathbf{k}_1} } \\
\ket{ \phi_{n  \mathbf{k}_1 + \mathbf{q}_s} } & = \sum_{n_1}
   \braket{ \psi_{n_2 \mathbf{k}_1 + \mathbf{q}_s} | \phi_{n  \mathbf{k}_1 + \mathbf{q}_s} } 
   \ket{ \psi_{n_2 \mathbf{k}_1 + \mathbf{q}_s} }
\label{eq:unitaryrot}
\end{align}
The ``known'' and ``sought'' e-ph matrix elements are those calculated with $\psi$ functions, 
but by $\mathcal{S}^{-1}$ transformation we get the $\phi$ ones. Therefore, we have to undo the 
unitary transformation Eq.~\ref{eq:unitaryrot}.

\subsubsection{Undoing the unitary transformation}

Note that $\mathbf{k}, \mathbf{k}_1, \mathbf{k}_1^s$ belong to the same star and thus the
additional symlinks can be established (also for the $k+q$ side of the brakets):
\begin{align}
\mathcal{S}_0 \psi_{n \mathbf{k}_1^s} (\mathbf{r}) & = \psi_{n \mathbf{k}} (\mathbf{r}) \\
\mathcal{S}'_0 \psi_{n \mathbf{k}_2^s} (\mathbf{r}) & = \psi_{n \mathbf{k} + \mathbf{q}} (\mathbf{r}) 
\end{align}
Therefore, the unitary transformation coefficients are
\begin{align}
\braket{ \psi_{n_1 \mathbf{k}_1} | \phi_{n  \mathbf{k}_1} } &= 
   \braket{ \mathcal{S}_1 \psi_{n_1 \mathbf{k}_1^s} | \mathcal{S}^{-1} \psi_{n  \mathbf{k}} }  = \\
&=  \braket{ \mathcal{S}_1 \psi_{n_1 \mathbf{k}_1^s} | \mathcal{S}^{-1} \mathcal{S}_0  \psi_{n_1 \mathbf{k}_1^s} } = \\
&=  \bra{ \psi_{n_1 \mathbf{k}_1^s} }   \mathcal{S}_1^{-1} \mathcal{S}^{-1} \mathcal{S}_0 \ket{\psi_{n \mathbf{k}_1^s} }  \\
\braket{ \psi_{n_2 \mathbf{k}_2} | \phi_{n  \mathbf{k}_2} } &= 
  \bra{ \psi_{n_2 \mathbf{k}_2^s} }   \mathcal{S}_2^{-1} \mathcal{S}^{-1} \mathcal{S}'_0 \ket{\psi_{n \mathbf{k}_2^s} }
\end{align}


\subsection{Effect of potential derivative rotation}

Since  $R\mathbf{q_s} = \mathbf{q} + \mathbf{G}_q$, the rotated local potential is  
\begin{align}
\mathcal{S}^{-1} & \partial^\alpha_{s \mathbf{q}} V^{loc}(\mathbf{r})\mathcal{S} =
   \partial^\alpha_{s \mathbf{q}} V^{loc}(\{R|\tau\} \mathbf{r}) = \\
&= -i \sum_{\mathbf{G}} (q_\alpha + G_\alpha)
   \tilde v_s (\mathbf{q}+\mathbf{G}) e^{ i(\mathbf{q}+\mathbf{G}) \cdot (R\mathbf{r}-\tau) } = \\
&= -i\sum_{\mathbf{G}} (R\mathbf{q_s} + \mathbf{G} - \mathbf{G}_q)_\alpha 
   \tilde v_s (R\mathbf{q_s} + \mathbf{G} - \mathbf{G}_q) 
   e^{ i( R\mathbf{q_s} + \mathbf{G} - \mathbf{G}_q ) \cdot (R\mathbf{r}-\tau) } = \\
&= -i\sum_{\mathbf{G}} (R\mathbf{q_s} + \mathbf{G})_\alpha 
   \tilde v_s (R\mathbf{q_s} + \mathbf{G}) 
   e^{ i( R \mathbf{q_s} + \mathbf{G}) \cdot (R\mathbf{r}-\tau) } 
\end{align}
where we have changed the sum over $G$ to $G-G_q$, and now we are going to change it to a sum over $R^{-1}G$.
We also write explicitly the Fourier components for the $s$-th atom, which is at position $\tau_s$ in the unit cell, 
according to Haritz's Eq.~(40) (square brakets):
\begin{align}
&= -i \sum_{\mathbf{G}} ( R(\mathbf{q_s} + R^{-1}\mathbf{G}) )_\alpha 
   \tilde v_s ( R(\mathbf{q_s} + R^{-1}\mathbf{G}) ) 
   e^{ i(\mathbf{q_s} + R^{-1}\mathbf{G}) \cdot ( \mathbf{r}-R^{-1}\tau)}  = \\
&= -i \sum_{\mathbf{G}} ( R(\mathbf{q_s} + \mathbf{G}) )_\alpha 
   \tilde v_s ( R(\mathbf{q_s} + \mathbf{G}) ) 
   e^{ i(\mathbf{q_s} + \mathbf{G}) \cdot ( \mathbf{r}-R^{-1}\tau)}  = \\
&= -i \sum_{\mathbf{G}} \sum_\beta R^{cart}_{\alpha\beta}(\mathbf{q_s} + \mathbf{G})_\beta 
   \Big[ e^{-i R(\mathbf{q_s} + \mathbf{G}) \cdot \tau_s } \frac{1}{\Omega}
   \int \mathrm{d}^3 \mathbf{r}' v_s(\mathbf{r}') e^{-i R(\mathbf{q_s} + \mathbf{G}) \cdot \mathbf{r}'}  \Big]
   e^{i (\mathbf{q_s} + \mathbf{G}) \cdot ( \mathbf{r}-R^{-1}\tau)}  = \\
&= -i \sum_\beta \sum_{\mathbf{G}} R^{cart}_{\alpha\beta}(\mathbf{q_s} + \mathbf{G})_\beta 
   e^{-i (\mathbf{q_s} + \mathbf{G}) \cdot R^{-1}( \tau_s + \tau) } 
   \Big[ \frac{1}{\Omega}
   \int \mathrm{d}^3 \mathbf{r}' v_s (\mathbf{r}') e^{-i (\mathbf{q_s} + \mathbf{G}) \cdot R^{-1}\mathbf{r}' }  \Big]
   e^{i (\mathbf{q_s} + \mathbf{G}) \cdot \mathbf{r} }
\end{align}
Since $v_s$ is the isolated atom pseudopotential, it is rotationally invariant, so doing the change 
$\mathbf{r}'' = R^{-1} \mathbf{r}'$ the integral term becomes
\begin{align}
 \frac{|R|}{\Omega} \int \mathrm{d}^3 \mathbf{r}''  v_s (\mathbf{r}'') e^{-i (\mathbf{q_s} + \mathbf{G}) \cdot \mathbf{r}'' } 
\end{align}
We take the $s'$-th related to $s$ by the symmetry operation $\mathcal{S}^{-1}$ as 
\begin{align}
\tau_{s'} = R^{-1}(\tau_s + \tau) + \mathbf{L}
\end{align} 
where $\mathbf{L}$ is a direct lattice vector. Since they are the same species, $v_s(\mathbf{r}) = v_{s'}(\mathbf{r})$.
Substituting above, we obtain $\tilde v_{s'} (\mathbf{q_s} + \mathbf{G})$ 
coefficients multiplied by the extra phase factor 
$e^{-i (\mathbf{q_s} + \mathbf{G}) \cdot \mathbf{L} } = e^{-i \mathbf{q_s} \cdot \mathbf{L} }$. Finally
\begin{align}
\mathcal{S}^{-1} & \partial^\alpha_{s \mathbf{q}} V^{loc}(\mathbf{r})\mathcal{S} = \\
 &= -i |R| \sum_\beta \sum_{\mathbf{G}} R^{cart}_{\alpha\beta}(\mathbf{q_s} + \mathbf{G})_\beta
   e^{-i \mathbf{q_s} \cdot \mathbf{L} }  \tilde v_{s'} (\mathbf{q_s} + \mathbf{G}) 
   e^{i (\mathbf{q_s} + \mathbf{G}) \cdot \mathbf{r} } = \\
 &=  -i |R| e^{-i \mathbf{q_s} \cdot \mathbf{L} }   \sum_\beta R^{cart}_{\alpha\beta}  
    \partial^\beta_{s' \mathbf{q}_s} V^{loc}(\mathbf{r})
\end{align}


\subsection{Full transformation}

Putting together the wavefunction and operator transformations we get
\begin{align}
\mathcal{S}^{-1} & \partial^\alpha_{s \mathbf{q}} V^{loc}(\mathbf{r})\mathcal{S} = 
   \partial^\alpha_{s \mathbf{q}} V^{loc}(\{R|\tau\} \mathbf{r})  = \\
&= -i|R|  e^{-i \mathbf{q_s} \cdot \mathbf{L} }  
   \sum_{n_1m_1}  
\bra{ \psi_{n \mathbf{k}_2^s} }  \mathcal{S'}_0^{-1} \mathcal{S} \mathcal{S}_2 \ket{\psi_{n_1 \mathbf{k}_2^s} }
\bra{ \psi_{m_1 \mathbf{k}_1^s} }  \mathcal{S}_1^{-1} \mathcal{S}^{-1} \mathcal{S}_0 \ket{\psi_{m \mathbf{k}_1^s} }
   \sum_\beta R^{cart}_{\alpha\beta}  \nonumber\\
&   \bra{\psi_{n_1 \mathbf{k}_1 + \mathbf{q}_s}}
       \partial^\beta_{s' \mathbf{q}_s} V^{loc}(\mathbf{r})
   \ket{\psi_{m_1 \mathbf{k}_1}}
\end{align}
where $\mathbf{q}$ is symlinked to $\mathbf{q}_s$ by $\mathcal{S} = \{R | \tau \}$,
$\mathbf{k} + \mathbf{q}$ is symlinked to $\mathbf{k}_2^s$ by $\mathcal{S'}_0$,
$\mathbf{k}$ is symlinked to $\mathbf{k}_1^s$ by $\mathcal{S}_0$, 
$\mathbf{k}_1 + \mathbf{q}_1$ is symlinked to $\mathbf{k}_2^s$ by $\mathcal{S}_2$,
$\mathbf{k}_1$ is symlinked to $\mathbf{k}_1^s$ by $\mathcal{S}_1$,
and the atoms are transformed as $\tau_s'=\mathcal{S}^{-1}\tau_s$.



\textcolor{blue}{TODO See what happens if $\mathcal{S}$ involves TR, spinors and non-local terms.}

%======================================================

\section{Rotation of $M_{mn}$}

This is supposed to serve as a test for phase fixing in transformation of matrix elements, as only
the overlap of wavefunctions is involved. We start from the usual definition
\begin{align}
M_{mn}(\mathbf{k},\mathbf{b}) \equiv \braket{u_{m\mathbf{k}}|u_{n\mathbf{k}+\mathbf{b}}}
\end{align}
at $\mathbf{k} \in \mathcal{M}_f$. We want to write these overlaps by transformation of 
$M_{mn}(\mathbf{k}^s,\mathbf{b})$ with $\mathbf{k}^s \in \mathcal{M}_i$, such that 
$\mathbf{k}$ and $\mathbf{k}^s$ are symlinked by $\mathcal{S}$ as
\begin{align}
\mathcal{S} \psi_{m \mathbf{k}^s} (\mathbf{r}) & = \psi_{m \mathbf{k}} (\mathbf{r})
\end{align}

We start without TR, taking $\mathcal{S} = \{R|\tau\}$. Therefore, 
\begin{align}
R \mathbf{k}_s & = \mathbf{k} + \mathbf{G}_s \\
R^{-1} (\mathbf{k}+\mathbf{b}) & = \mathbf{k}_s + R^{-1} \mathbf{b} - R^{-1} \mathbf{G}_s 
\end{align}
which will be useful later.
\begin{align}
M_{mn}(\mathbf{k},\mathbf{b}) & = \braket{e^{-i\mathbf{k}\cdot\mathbf{r}} \psi_{m\mathbf{k}} | 
    e^{-i(\mathbf{k}+\mathbf{b})\cdot\mathbf{r}} \psi_{n\mathbf{k}+\mathbf{b}}} = \\
&=  \braket{\psi_{m\mathbf{k}} | e^{-i\mathbf{b}\cdot\mathbf{r}}|\psi_{n\mathbf{k}+\mathbf{b}}} = \\
&=  \braket{\mathcal{S} \psi_{m\mathbf{k}_s} |  e^{-i\mathbf{b}\cdot\mathbf{r}}|\psi_{n\mathbf{k}+\mathbf{b}}} = \\
&=  \braket{ \psi_{m\mathbf{k}_s} | \mathcal{S}^{-1} e^{-i\mathbf{b}\cdot\mathbf{r}} \mathcal{S} 
            \mathcal{S}^{-1} | \psi_{n\mathbf{k}+\mathbf{b}}} = \\
&=  \braket{ \psi_{m\mathbf{k}_s} | e^{-i\mathbf{b}\cdot\mathcal{S}\mathbf{r}} | 
            \mathcal{S}^{-1} \psi_{n\mathbf{k}+\mathbf{b}}} = \\
&=  \braket{ \psi_{m\mathbf{k}_s} | e^{-i\mathbf{b}\cdot(R\mathbf{r}-\tau)} | \phi_{n\mathbf{k}_s+R^{-1}\mathbf{b}}} = \\
&=  e^{i\mathbf{b}\cdot\tau} \braket{ \psi_{m\mathbf{k}_s} | e^{-iR^{-1}\mathbf{b}\cdot\mathbf{r}} | 
           \phi_{n\mathbf{k}_s+R^{-1}\mathbf{b}}} 
\end{align}
Here we have defined the inverse transformation of $\psi_{n \mathbf{k}+\mathbf{b}}$ 
(using the $\phi$ notation because it is not a symlink) as
\begin{align}
\ket{ \mathcal{S}^{-1} \psi_{n \mathbf{k} + \mathbf{b}} }  & \equiv \ket { \phi_{n \mathbf{k}_s+R^{-1}\mathbf{b}} }  = \\
&= \sum_{n_1} \braket{ \psi_{n_1 \mathbf{k}_s+R^{-1}\mathbf{b} } | \phi_{n \mathbf{k}_s+R^{-1}\mathbf{b}} } 
      \ket{ \psi_{n_1 \mathbf{k}_s+R^{-1}\mathbf{b} } } = \\
&= \sum_{n_1} \braket{ \psi_{n_1 \mathbf{k}_1^s} | \mathcal{S}_1^{-1} \mathcal{S}^{-1} \mathcal{S}_0 | \psi_{n \mathbf{k}_1^s} } 
      \ket{ \psi_{n_1 \mathbf{k}_s+R^{-1}\mathbf{b} } }
\end{align}
where we have used the symlinks $\mathcal{S}_1$ and $\mathcal{S}_0$, respectively, 
of wavevectors $\mathbf{k}_s+R^{-1}\mathbf{b}$ and $\mathbf{k}+\mathbf{b}$, 
which belong to the same start, to the wavevector $\mathbf{k}_1^s \in \mathcal{M}_i$. Therefore,
\begin{align}
M_{mn}(\mathbf{k},\mathbf{b}) & = e^{i\mathbf{b}\cdot\tau} \sum_{n_1} 
  \braket{ \psi_{n_1 \mathbf{k}_s+R^{-1}\mathbf{b} } | \phi_{n \mathbf{k}_s+R^{-1}\mathbf{b}} } 
  \braket{ \psi_{m\mathbf{k}_s} | e^{-iR^{-1}\mathbf{b}\cdot\mathbf{r}} | \psi_{n_1 \mathbf{k}_s+R^{-1}\mathbf{b} } } = \\
&= e^{i\mathbf{b}\cdot\tau} \sum_{n_1}
  \braket{ \psi_{n_1 \mathbf{k}_1^s} | \mathcal{S}_1^{-1} \mathcal{S}^{-1} \mathcal{S}_0 | \psi_{n \mathbf{k}_1^s} }
  M_{m n_1}(\mathbf{k}_s, R^{-1}\mathbf{b})
\end{align}

With TR we apply $\mathcal{S} = \mathcal{T} \{ R|\tau \}$. Therefore 
\begin{align}
-R \mathbf{k}_s & = \mathbf{k} + \mathbf{G}_s \\
-R^{-1}(\mathbf{k} + \mathbf{b}) & = \mathbf{k}_s - R^{-1}\mathbf{b} + R^{-1} \mathbf{G}_s \\
\mathcal{S}^{-1} e^{-i \mathbf{b} \cdot \mathbf{r}} \mathcal{S} & = 
 \Big( e^{-i \mathbf{b} \cdot \{R|\tau\} \mathbf{r}} \Big)^* = 
  e^{i \mathbf{b} \cdot \{R|\tau\} \mathbf{r}}  \\
\ket{ \mathcal{S}^{-1} \psi_{n \mathbf{k} + \mathbf{b}} }  & = \ket { \phi_{n \mathbf{k}_s-R^{-1}\mathbf{b}} } \\
M_{mn}(\mathbf{k},\mathbf{b}) & = e^{-i\mathbf{b}\cdot\tau} \sum_{n_1}
  \braket{ \psi_{n_1 \mathbf{k}_s-R^{-1}\mathbf{b} } | \phi_{n \mathbf{k}_s-R^{-1}\mathbf{b}} }
  M_{m n_1}(\mathbf{k}_s, -R^{-1}\mathbf{b})
\end{align}


\textcolor{blue}{TODO see what happens for spinors}


\bibliography{intw_sym}

\end{document}
