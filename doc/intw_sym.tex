\documentclass[aps,prb,preprint,superscriptaddress,showpacs,showkeys]{revtex4-2}  
%\documentclass[aps,prb,twocolumn,superscriptaddress,showpacs,showkeys]{revtex4-2}  
\bibliographystyle{apsrev4-2}

\usepackage[pdftex]{graphicx}
\usepackage{color}
\usepackage{amsmath}
\usepackage{multirow}
\usepackage{placeins}
\usepackage{braket}

\usepackage{changes}
%\usepackage[final]{changes}

\usepackage{xr}
%\externaldocument[SM-]{sm_dmi_v2}

\begin{document}

% Use the \preprint command to place your local institutional report
% number in the upper righthand corner of the title page in preprint mode.
% Multiple \preprint commands are allowed.
% Use the 'preprintnumbers' class option to override journal defaults
% to display numbers if necessary
%\preprint{}

%Title of paper
\title{Summary of symmetries in INTW matrix elements}



% repeat the \author .. \affiliation etc. as needed
% \email, \thanks, \homepage, \altaffiliation all apply to the current
% author. Explanatory text should go in the []'s, actual e-mail
% address or url should go in the {}'s for \email and \homepage.
% Please use the appropriate macro foreach each type of information

% \affiliation command applies to all authors since the last
% \affiliation command. The \affiliation command should follow the
% other information
% \affiliation can be followed by \email, \homepage, \thanks as well.

\newcommand{\QUIMI}[0]{{
Departamento de Pol\'{\i}meros y Materiales Avanzados: 
F\'{\i}sica, Qu\'{\i}mica y Tecnolog\'{\i}a, Facultad de Qu\'{\i}mica UPV/EHU,
Apartado 1072, 20080 Donostia-San Sebasti\'an, Spain}}

\newcommand{\DIPC}[0]{{
Donostia International Physics Center, 
Paseo Manuel de Lardiz\'abal 4, 20018 Donostia-San Sebasti\'an, Spain}}

\newcommand{\CFM}[0]{{
Centro de F\'{\i}sica de Materiales CFM/MPC (CSIC-UPV/EHU), 
Paseo Manuel de Lardiz\'abal 5, 20018 Donostia-San Sebasti\'an, Spain}}

%\author{M. Blanco-Rey}
%\affiliation{\QUIMI}
%\affiliation{\DIPC}
%\affiliation{\CFM}


\date{\today}

%\begin{abstract}
%bla bla
%\end{abstract}

\maketitle

%======================================================


\section{Notation}

We consider $\mathbf{k} \in \mathcal{M}_f$ to be the wavevectors in the full mesh of the 1BZ and 
$\mathbf{k}_s \in \mathcal{M}_i$ to be the wavevectors in the irreducible wedge of the 1BZ.

The symmetry elements are $\mathcal{S} = \{R | \mathbf{\tau} \} = \mathcal{R} $ without time reversal (TR)
or $\mathcal{S} = \mathcal{T} \{R| \mathbf{\tau} \} = \mathcal{TR} $ with 
time reversal, where $\{R|\tau \}$ is an element of the space group of the crystal such that it 
transforms positions as:
\begin{itemize}
\item $\mathbf{r}' = R \mathbf{r} - \mathbf{\tau} $ (conventional notation)
\item $\mathbf{r}' = R (\mathbf{r} + \mathbf{\tau} )$ (QE notation)
\end{itemize}

The $\mathbf{k} \in \mathcal{M}_f$ are linked to the $\mathbf{k}_s \in \mathcal{M}_i$ vectors 
by symmetry operations. We say they are ``symlinked'' by an operation $\mathcal{S}$ when 
we can write $R \mathbf{k_s} = \mathbf{k} + \mathbf{G}_s$ with TR or 
$-R \mathbf{k_s} = \mathbf{k} + \mathbf{G}_s$ without TR, 
where $\mathbf{G}_s$ is a vector of the crystal reciprocal lattice. 

The choice of an $\mathcal{S}$ operation to connect a $\mathbf{k}$ and a $\mathbf{k}_s$ may not be unique. 
In the INTW implementation we nevertheless set a given symlink, which will fix the 
wavefunction phase.

Transformation by $\mathcal{S}=\{ R|\tau \}$ of a real-space vector, a wavefunction and 
an operator 
\begin{itemize}
\item in the conventional notation:
\begin{align}
\mathbf{r}' & = \mathcal{S}^{-1} \mathbf{r} \mathcal{S} \equiv \mathcal{S}\mathbf{r}
= \{ R|\tau \}  \mathbf{r} = R  \mathbf{r} - \tau \\
  \psi'( \mathbf{r} ) & = \psi ( \mathcal{S}^{-1} \mathbf{r} ) = \psi ( \{ R^{-1}| -R^{-1}\tau \}  \mathbf{r} ) =
  \psi (R^{-1}(\mathbf{r}+\tau))  \\
\mathcal{O}'( \mathbf{r} ) &= \mathcal{S}^{-1} \mathcal{O}( \mathbf{r} ) \mathcal{S} =
\mathcal{O}( \mathcal{S}^{-1} \mathbf{r} \mathcal{S} ) \equiv
\mathcal{O}( \mathcal{S}\mathbf{r}) = \mathcal{O}(R  \mathbf{r} - \tau)
\end{align}
\item in the QE notation:
\begin{align}
\mathbf{r}' & = \mathcal{S}^{-1} \mathbf{r} \mathcal{S} \equiv \mathcal{S}\mathbf{r} 
= \{ R|\tau \}  \mathbf{r} = R (\mathbf{r} + \tau) \\
  \psi'( \mathbf{r} ) & = \psi ( \mathcal{S}^{-1} \mathbf{r} ) = \psi ( \{ R^{-1}| -R\tau \}  \mathbf{r} ) = 
  \psi (R^{-1}\mathbf{r} - \tau)  \\
\mathcal{O}'( \mathbf{r} ) &= \mathcal{S}^{-1} \mathcal{O}( \mathbf{r} ) \mathcal{S} = 
\mathcal{O}( \mathcal{S}^{-1} \mathbf{r} \mathcal{S} ) \equiv 
\mathcal{O}( \mathcal{S}\mathbf{r}) = \mathcal{O}(R (\mathbf{r} + \tau))
\end{align}
\end{itemize}

%======================================================

\section{Rotation of wavefunctions}

By Bloch's theorem, we write the wavefunctions as 
\begin{align}
\psi_{n\mathbf{k}} & = e^{i\mathbf{k}\cdot\mathbf{r}} u_{n\mathbf{k}} (\mathbf{r}) \\
u_{n\mathbf{k}} (\mathbf{r}) & = \sum_{\mathbf{G}} e^{i\mathbf{G}\cdot\mathbf{r}} 
     \tilde u_{n\mathbf{k}} (\mathbf{G})
\end{align}
where the sum runs over reciprocal lattice vectors.

Without TR, we have $R \mathbf{k_s} = \mathbf{k} + \mathbf{G}_s$. Following the QE notation:
\begin{align}
\{R| \mathbf{\tau} \} \psi_{n \mathbf{k_s}}(\mathbf{r}) & = \psi_{n \mathbf{k_s}}(R^{-1}\mathbf{r}-\mathbf{\tau})  = \\
& = e^{i\mathbf{k}_s \cdot (R^{-1}\mathbf{r}-\mathbf{\tau})} u_{n\mathbf{k}_s} (R^{-1}\mathbf{r}-\mathbf{\tau}) = \\
& = e^{i\mathbf{k}_s \cdot (R^{-1}\mathbf{r}-\mathbf{\tau})} 
   \sum_{\mathbf{G}} e^{i\mathbf{G}\cdot (R^{-1}\mathbf{r}-\mathbf{\tau})} 
   \tilde u_{n\mathbf{k}_s} (\mathbf{G}) = \\
& = e^{i R \mathbf{k}_s \cdot \mathbf{r}} e^{-i \mathbf{k}_s \cdot \tau} 
   \sum_{\mathbf{G}}  e^{i R \mathbf{G}\cdot \mathbf{r}} e^{-i \mathbf{G}\cdot\mathbf{\tau} }
   \tilde u_{n\mathbf{k}_s} (\mathbf{G}) = \\
& = e^{i \mathbf{k} \cdot \mathbf{r} } e^{-i\mathbf{k}_s \cdot \mathbf{\tau}}
   \sum_{\mathbf{G}} e^{i ( R \mathbf{G} + {\mathbf{G}_s} ) \cdot \mathbf{r} }
   e^{-i \mathbf{G}\cdot\mathbf{\tau} } \tilde u_{n\mathbf{k}_s} (\mathbf{G}) = \\
& = e^{i \mathbf{k} \cdot \mathbf{r} } u_{n\mathbf{k}} (\mathbf{r}) = \psi_{n\mathbf{k}} (\mathbf{r}) 
\label{eq:rot_noTR1}
\end{align}
where
\begin{align}
u_{n\mathbf{k}} (\mathbf{r}) & =  \sum_{\mathbf{G}'} e^{i\mathbf{G}' \cdot \mathbf{r}} \tilde u_{n\mathbf{k}} (\mathbf{G'}) \\
\tilde u_{n\mathbf{k}} (\mathbf{G}') & = e^{- i \mathbf{k}_s \cdot \mathbf{\tau}} 
   e^{-i R^{-1}( \mathbf{G}' - \mathbf{G}_s) \cdot \mathbf{\tau}}
  \tilde u_{n\mathbf{k}_s} ( R^{-1}( \mathbf{G}' - \mathbf{G}_s) ) 
%\tilde u_{n\mathbf{k}} (R \mathbf{G} + {\mathbf{G}_s} ) & = e^{- i \mathbf{k}_s \cdot \mathbf{\tau}}  
%   e^{-i \mathbf{G} \cdot \mathbf{\tau}} \tilde u_{n\mathbf{k}_s} (\mathbf{G}) 
\label{eq:rot_noTR2}
\end{align}

With TR, we have $-R \mathbf{k_s} = \mathbf{k} + \mathbf{G}_s$:
\begin{align}
\mathcal{T} \{R| \mathbf{\tau} \} \psi_{n \mathbf{k_s}}(\mathbf{r}) & =  \Big[ \{R| \mathbf{\tau} \} \psi_{n \mathbf{k_s}}(\mathbf{r}) \Big]^* = \\
& = \Big[  e^{i R \mathbf{k}_s \cdot \mathbf{r}} e^{-i \mathbf{k}_s \cdot \tau}
   \sum_{\mathbf{G}}  e^{i R \mathbf{G}\cdot \mathbf{r}} e^{-i \mathbf{G}\cdot\mathbf{\tau} }
   \tilde u_{n\mathbf{k}_s} (\mathbf{G}) \Big]^* = \\
& = e^{i (\mathbf{k} + \mathbf{G}_s) \cdot \mathbf{r}} e^{i \mathbf{k}_s \cdot \mathbf{\tau} } \sum_{\mathbf{G}}  
    e^{ - i R \mathbf{G}\cdot \mathbf{r}}  e^{i \mathbf{G} \cdot \mathbf{\tau} } \tilde u_{n\mathbf{k}_s}^* (\mathbf{G}) = \\
& = e^{i \mathbf{k} \cdot \mathbf{r} } u_{n\mathbf{k}} (\mathbf{r}) = \psi_{n\mathbf{k}} (\mathbf{r}) 
\end{align}
where 
\begin{align}
u_{n\mathbf{k}} (\mathbf{r}) & =  \sum_{\mathbf{G'}} e^{i\mathbf{G'}\cdot\mathbf{r}} \tilde u_{n\mathbf{k}} (\mathbf{G'}) \\
\tilde u_{n\mathbf{k}} (\mathbf{G'}) & = e^{i \mathbf{k}_s \cdot \mathbf{\tau} } 
   e^{i R^{-1}( \mathbf{G}_s - \mathbf{G}') \cdot \mathbf{\tau} }  
  \tilde u_{n\mathbf{k}_s}^* ( R^{-1}( \mathbf{G}_s - \mathbf{G}') )
\end{align}

{\it Note}: the original routine rotate\_wfc\_test is implemented without 
the global phase factor $e^{- i \mathbf{k}_s \cdot \mathbf{\tau}}$.


As Lax states on page 293, the TR and point (or space) group operations commute. 
Indeed, in the equations above, we see that 
\begin{align}
\mathcal{T} \{R| \mathbf{\tau} \} \psi_{n \mathbf{k_s}}(\mathbf{r}) =  
    \Big[ \{R| \mathbf{\tau} \} \psi_{n \mathbf{k_s}}(\mathbf{r}) \Big]^* = 
\{R| \mathbf{\tau} \} \psi^*_{n \mathbf{k_s}}(\mathbf{r}) = 
    \{R| \mathbf{\tau} \} \mathcal{T} \psi_{n \mathbf{k_s}}(\mathbf{r})
\end{align}


\subsection{Spinor rotation}

This commutation rule is true also in 
the case of spinors. We write the spinor wavefunction as 
\begin{align}
\psi_{n\mathbf{k}}(\mathbf{r}) = \psi^\uparrow_{n\mathbf{k}}(\mathbf{r}) \ket{\uparrow} + 
  \psi^\downarrow_{n\mathbf{k}}(\mathbf{r}) \ket{\downarrow} \equiv 
\begin{pmatrix}
\psi^\uparrow_{n\mathbf{k}}(\mathbf{r}) \\
\psi^\downarrow_{n\mathbf{k}}(\mathbf{r}) 
\end{pmatrix}
\end{align}
Here, the transformations apply independently on the $\mathbf{r}$-dependent coordinates 
and spin-$\frac{1}{2}$ states. On the latter, the improper part and the fractional translation of
$\{R| \mathbf{\tau} \}$ have no effect. If we call $R^{(0)}$ to the pure rotation component of that 
transformation, consisting of a counterclockwise rotation of an $\alpha$ angle around direction 
$\hat n = (n_x, n_y, n_z)$, the rotated spin states will be given by the $2 \times 2$ matrix
\begin{align}
D_{1/2}(R^{(0)}) & = \cos\frac{\alpha}{2} \sigma_0 - i \sin\frac{\alpha}{2} \mathbf{\sigma}\cdot\hat{n} = \\
& = \begin{pmatrix}
\cos\frac{\alpha}{2} -i n_z \sin\frac{\alpha}{2}   &   -\sin\frac{\alpha}{2} (n_y + i n_x ) \\
-\sin\frac{\alpha}{2} (n_y -i n_x )  &  \cos\frac{\alpha}{2} +i n_z \sin\frac{\alpha}{2} \\
\end{pmatrix}
\label{eq:Srot}
\end{align}
applied on $\ket{\uparrow}=\binom{1}{0}$ and $\ket{\downarrow}=\binom{0}{1}$, where
$\sigma_0$ is the $2 \times 2$ unit matrix and $\sigma_{x,y,z}$ are the Pauli matrices.

\textcolor{red}{NOTE: I am writing above the transformation of the spinor by multiplying this matrix on the 
vector column of spinor components. Instead, in INTW it is implemented with 
the proper part of the inverse $\{R| \mathbf{\tau} \}^{-1}$, and that seems to work.
As Asier says, the spin is like another variable, 
so it should go with the inverse $(R^{(0)})^{-1}$.}


\textcolor{red}{HOWEVER, I have a big DOUBT with this... when spinors are introduced, the suitable 
symmetries are those of the {\bf double group} (DG), i.e. we have to include rotations by $2\pi$ which will invert the spin.
In practice, this adds a global phase $e^{i\pi}$ to the wavefunction, so it will not have an effect on most
applications, but it may have an effect on the rotation of matrix elements, I think. 
In addition, there is another issue with a broader effect on INTW: the use of the inverse\_indices array, 
which is constructed without considering the spin rotation. 
Asier rotates the spins by calculating the angle $\alpha$ and axis $\hat n$ of the symmetry element, with 
the convention that the rotation is anti-clockwise and the angle lies in $[-\pi,\pi]$. 
I have observed for the cases with $\alpha=\pi$ (and so far not with other rotations), 
where the rotation element is idempotent on the $\mathbf{r}$ degree
of freedom, the property $R^{(0)}(R^{(0)})^{-1}=E$ ($E$ is the identity) is not meet for the spin part. 
This needs to be taken into account when rotating matrix elements for spinors, but I am not sure how.
I am also worried about the effect of the DG symmetry in other parts of the code.
}


With both point or space group element and TR, the spinor transforms as 
\begin{align}
\mathcal{T} \{R| \mathbf{\tau} \} \psi_{n\mathbf{k}}(\mathbf{r}) & = 
\{R| \mathbf{\tau} \} \mathcal{T} \psi_{n\mathbf{k}}(\mathbf{r}) =  \\
&= D_{1/2}(R^{(0)}) (i\sigma_y) 
   \Big( [ \psi^\uparrow_{n\mathbf{k}}(R^{-1}\mathbf{r}-\mathbf{\tau})) ]^*  \ket{\uparrow} +
         [ \psi^\downarrow_{n\mathbf{k}}(R^{-1}\mathbf{r}-\mathbf{\tau})) ]^* \ket{\downarrow}  \Big) = \\
&= [ \psi^\uparrow_{n\mathbf{k}}(R^{-1}\mathbf{r}-\mathbf{\tau})) ]^*  
   D_{1/2}(R^{(0)}) (i\sigma_y) \ket{\uparrow} +
   [ \psi^\downarrow_{n\mathbf{k}}(R^{-1}\mathbf{r}-\mathbf{\tau})) ]^* 
   D_{1/2}(R^{(0)}) (i\sigma_y) \ket{\downarrow} = \\
&= - [ \psi^\uparrow_{n\mathbf{k}}(R^{-1}\mathbf{r}-\mathbf{\tau})) ]^*  
   D_{1/2}(R^{(0)}) \ket{\downarrow} +
   [ \psi^\downarrow_{n\mathbf{k}}(R^{-1}\mathbf{r}-\mathbf{\tau})) ]^* 
   D_{1/2}(R^{(0)}) \ket{\uparrow} 
\end{align}
where $i\sigma_y$ is the TR operator for the spin degree of freedom.




%======================================================

\section{Rotation of $M_{mn}$}

This is supposed to serve as a test for phase fixing in transformation of matrix elements, as only
the overlap of wavefunctions is involved. We start from the usual definition
\begin{align}
M_{mn}(\mathbf{k},\mathbf{b}) \equiv \braket{u_{m\mathbf{k}}|u_{n\mathbf{k}+\mathbf{b}}}
\end{align}
at $\mathbf{k} \in \mathcal{M}_f$. We want to write these overlaps by transformation of
$M_{mn}(\mathbf{k}_s,\mathbf{b})$ with $\mathbf{k}_s \in \mathcal{M}_i$, such that
$\mathbf{k}$ and $\mathbf{k}^s$ are symlinked by $\mathcal{S}$ as
\begin{align}
\mathcal{S} \psi_{m \mathbf{k}_s} (\mathbf{r}) & = \psi_{m \mathbf{k}} (\mathbf{r})
\end{align}

We start without TR, taking $\mathcal{S} = \{R|\tau\}$. Therefore,
\begin{align}
R \mathbf{k}_s & = \mathbf{k} + \mathbf{G}_s \\
R^{-1} (\mathbf{k}+\mathbf{b}) & = \mathbf{k}_s + R^{-1} \mathbf{b} - R^{-1} \mathbf{G}_s
\end{align}
which will be useful later.
\begin{align}
M_{mn}(\mathbf{k},\mathbf{b}) & = \braket{e^{-i\mathbf{k}\cdot\mathbf{r}} \psi_{m\mathbf{k}} |
    e^{-i(\mathbf{k}+\mathbf{b})\cdot\mathbf{r}} \psi_{n\mathbf{k}+\mathbf{b}}} = \\
&=  \braket{\psi_{m\mathbf{k}} | e^{-i\mathbf{b}\cdot\mathbf{r}}|\psi_{n\mathbf{k}+\mathbf{b}}} = \\
&=  \braket{\mathcal{S} \psi_{m\mathbf{k}_s} |  e^{-i\mathbf{b}\cdot\mathbf{r}}| 
            \mathcal{S} \mathcal{S}^{-1} \psi_{n\mathbf{k}+\mathbf{b}}} = \\
&= \braket{\mathcal{S} \psi_{m\mathbf{k}_s} |  e^{-i\mathbf{b}\cdot\mathbf{r}}| 
            \mathcal{S} \phi_{n\mathbf{k}_s+R^{-1}\mathbf{b}}}
\end{align}
Here we have defined the inverse transformation of $\psi_{n \mathbf{k}+\mathbf{b}}$
using the $\phi$ notation to distinguish it from the wavefunctions $\psi$ obtained by {\it symlink}
with the same wavevector. They are related by 
\begin{align}
\ket{ \mathcal{S}^{-1} \psi_{n \mathbf{k} + \mathbf{b}} }  & \equiv \ket { \phi_{n \mathbf{k}_s+R^{-1}\mathbf{b}} }  = \\
&= \sum_{n_1} \braket{ \psi_{n_1 \mathbf{k}_s+R^{-1}\mathbf{b} } | \phi_{n \mathbf{k}_s+R^{-1}\mathbf{b}} }
      \ket{ \psi_{n_1 \mathbf{k}_s+R^{-1}\mathbf{b} } }
\end{align}
This would be Evarestov eq. 3.8.12 and Lax eq. 3.7.1 for the particular case of the wavevector left 
invariant by the rotation.
In the case of spinors, each component transforms as
\begin{align}
\ket { \phi_{n \mathbf{k}_s+R^{-1}\mathbf{b}}^\sigma }  & = 
   \sum_{n_1} \big( 
      \braket{ \psi_{n_1 \mathbf{k}_s+R^{-1}\mathbf{b} }^\uparrow | \phi_{n \mathbf{k}_s+R^{-1}\mathbf{b}}^\uparrow } 
    + \braket{ \psi_{n_1 \mathbf{k}_s+R^{-1}\mathbf{b} }^\downarrow | \phi_{n \mathbf{k}_s+R^{-1}\mathbf{b}}^\downarrow }  
    \big)
    \ket{ \psi_{n_1 \mathbf{k}_s+R^{-1}\mathbf{b} }^\sigma  } = \\
 &=  \sum_{n_1} \braket{ \psi_{n_1 \mathbf{k}_s+R^{-1}\mathbf{b} } | \phi_{n \mathbf{k}_s+R^{-1}\mathbf{b}} }
    \ket{ \psi_{n_1 \mathbf{k}_s+R^{-1}\mathbf{b} }^\sigma  }
,\quad \sigma = \uparrow, \downarrow
\end{align}
\textcolor{red}{Creo que esto es así, pero no estoy del todo segura.}

We will deal with the calculation of the coefficients later.
We continue:
\begin{align}
M_{mn}(\mathbf{k},\mathbf{b}) & = \sum_{n_1} 
  \braket{ \psi_{n_1 \mathbf{k}_s+R^{-1}\mathbf{b} } | \phi_{n \mathbf{k}_s+R^{-1}\mathbf{b}} } 
  \braket{\mathcal{S} \psi_{m\mathbf{k}_s} |  e^{-i\mathbf{b}\cdot\mathbf{r}} | \mathcal{S} \psi_{n_1 \mathbf{k}_s+R^{-1}\mathbf{b} } } = \\
&= \sum_{n_1} 
  \braket{ \psi_{n_1 \mathbf{k}_s+R^{-1}\mathbf{b} } | \phi_{n \mathbf{k}_s+R^{-1}\mathbf{b}} }
  \int \mathrm{d}^3 \mathbf{r} \psi_{m\mathbf{k}_s}^* (R^{-1} \mathbf{r} - \tau) 
       e^{-i\mathbf{b}\cdot\mathbf{r}} 
       \psi_{n_1 \mathbf{k}_s +R^{-1}\mathbf{b}} (R^{-1} \mathbf{r} - \tau) = \\
&= e^{-iR^{-1}\mathbf{b} \cdot \tau} e^{i(k_s + R^{-1}\mathbf{b}) \cdot \mathbf{L}} \sum_{n_1}
  \braket{ \psi_{n_1 \mathbf{k}_s+R^{-1}\mathbf{b} } | \phi_{n \mathbf{k}_s+R^{-1}\mathbf{b}} }
  \int \mathrm{d}^3 \mathbf{r}' \psi_{m\mathbf{k}_s}^* ( \mathbf{r}' )  e^{-iR^{-1}\mathbf{b}\cdot\mathbf{r}'} 
       \psi_{n_1 \mathbf{k}_s +R^{-1}\mathbf{b}} ( \mathbf{r}' ) =  \\
&= e^{-iR^{-1}\mathbf{b} \cdot \tau} e^{i(k_s + R^{-1}\mathbf{b}) \cdot \mathbf{L}}  \sum_{n_1}
  \braket{ \psi_{n_1 \mathbf{k}_s+R^{-1}\mathbf{b} } | \phi_{n \mathbf{k}_s+R^{-1}\mathbf{b}} }
  M_{mn_1} (\mathbf{k}_s, R^{-1}\mathbf{b} )
\label{eq:Mmnrot}
\end{align}
where $\mathbf{L}$ is a lattice vector that accounts for the difference between the fractional 
translations of $\mathcal{S}$ and  $\mathcal{S}^{-1}$ elements:
\begin{align}
\mathcal{S} & = \{ R | \tau \}  \quad \rightarrow \quad \mathcal{S}^{-1} = \{ R^{-1} | -R\tau \} \equiv \{ R^{-1} | \tau^i \} \\
 \mathbf{L} & = \tau^i + R\tau
\label{eq:Sinv_tau} 
\end{align} 

Now, the case with TR, where $\mathcal{S} = \mathcal{T} \{ R|\tau \}$. Therefore
\begin{align}
-R \mathbf{k}_s & = \mathbf{k} + \mathbf{G}_s \\
-R^{-1}(\mathbf{k} + \mathbf{b}) & = \mathbf{k}_s - R^{-1}\mathbf{b} + R^{-1} \mathbf{G}_s \\
\ket{\mathcal{S} \psi_{n \mathbf{k}+\mathbf{b}}} & = \ket{\phi_{n \mathbf{k}_s-R^{-1}\mathbf{b}} }
\end{align}
When applying $\mathcal{S}$, we must conjugate the wavefunctions in the integral:
\begin{align}
M_{mn}(\mathbf{k},\mathbf{b}) & = 
  \sum_{n_1}
  \Big( \braket{ \psi_{n_1 \mathbf{k}_s-R^{-1}\mathbf{b} } | \phi_{n \mathbf{k}_s-R^{-1}\mathbf{b}} } \Big)^*
  \int \mathrm{d}^3 \mathbf{r} \psi_{m\mathbf{k}_s} (R^{-1} \mathbf{r} - \tau)
       e^{-i\mathbf{b}\cdot\mathbf{r}}
       \psi_{n_1 \mathbf{k}_s -R^{-1}\mathbf{b}}^* (R^{-1} \mathbf{r} - \tau) = \\
&= e^{-iR^{-1}\mathbf{b} \cdot \tau} e^{i(k_s + R^{-1}\mathbf{b}) \cdot \mathbf{L}} \sum_{n_1}
   \Big( \braket{ \psi_{n_1 \mathbf{k}_s-R^{-1}\mathbf{b} } | \phi_{n \mathbf{k}_s-R^{-1}\mathbf{b}} } \Big)^*  \nonumber\\
  & \int \mathrm{d}^3 \mathbf{r}' \psi_{m\mathbf{k}_s} (\mathbf{r}')
       e^{-iR^{-1}\mathbf{b}\cdot\mathbf{r}'}
       \psi_{n_1 \mathbf{k}_s -R^{-1}\mathbf{b}}^* ( \mathbf{r}' ) =  \\
&= e^{-iR^{-1}\mathbf{b} \cdot \tau}  e^{i(k_s + R^{-1}\mathbf{b}) \cdot \mathbf{L}}  \sum_{n_1}
   \Big( \braket{ \psi_{n_1 \mathbf{k}_s-R^{-1}\mathbf{b} } | \phi_{n \mathbf{k}_s-R^{-1}\mathbf{b}} } 
    M_{mn_1} (\mathbf{k}_s, -R^{-1}\mathbf{b} ) \Big)^* 
\label{eq:Mmnrot_tr}
\end{align}


With spinors, the $M_{mn}$ elements are
\begin{align}
M_{mn}(\mathbf{k},\mathbf{b}) & = \sum_{\sigma=\uparrow,\downarrow}  
     \braket{u_{m\mathbf{k}^\sigma}|u_{n\mathbf{k}+\mathbf{b}}^\sigma} = \\
& = \sum_{\sigma} \sum_{n_1} 
  \braket{ \psi_{n_1 \mathbf{k}_s+R^{-1}\mathbf{b} } | \phi_{n \mathbf{k}_s+R^{-1}\mathbf{b}} }
  \braket{ \big( \mathcal{S} \psi_{m\mathbf{k}_s} \big)^\sigma |  e^{-i\mathbf{b}\cdot\mathbf{r}} | 
           \big( \mathcal{S} \psi_{n_1 \mathbf{k}_s+R^{-1}\mathbf{b} }\big)^\sigma  } 
\end{align} 
where
\begin{align}
\ket{ \big( \mathcal{S} \psi_{m\mathbf{k}_s} \big)^\sigma } & = \sum_{\sigma'} d^{\sigma\sigma'} 
 \ket{ \mathcal{S} \psi_{m\mathbf{k}_s}^{\sigma'}} \ket{\sigma'}
\end{align}
Here, $d^{\sigma\sigma'}$ are the elements of the spin rotation matrix $D_{1/2}(R^{(0)})$ (Eq.~\ref{eq:Srot}).
Substituting:
\begin{align}
M_{mn}(\mathbf{k},\mathbf{b}) & =  \sum_{\sigma} \sum_{n_1}
  \braket{ \psi_{n_1 \mathbf{k}_s+R^{-1}\mathbf{b} } | \phi_{n \mathbf{k}_s+R^{-1}\mathbf{b}} }
  \sum_{\sigma'\sigma''} (d^{\sigma\sigma'})^* \bra{\sigma'} 
  \braket{ \mathcal{S} \psi_{m\mathbf{k}_s}^{\sigma'} |  e^{-i\mathbf{b}\cdot\mathbf{r}} |
           \mathcal{S} \psi_{n_1 \mathbf{k}_s+R^{-1}\mathbf{b} }^{\sigma''} } \ket{\sigma''} d^{\sigma\sigma''} 
\end{align}
Since 
\begin{align}
\sum_{\sigma} (d^{\sigma\sigma'})^* d^{\sigma\sigma''} = \delta_{\sigma'\sigma''}
\end{align}
we retrieve the same expression as Eq.~\ref{eq:Mmnrot} for the rotation of elements.

In the case of TR in $\mathcal{S}$, the $d^{\sigma\sigma'}$ coefficients are those of 
the $D_{1/2}(R^{(0)} i \sigma_y$ matrix.
As this unitarity property is kept with this matrix, we get the same as  Eq.~\ref{eq:Mmnrot_tr}.

\textcolor{red}{All of the above seems to work for all the tests I have tried except for SOC (Bi and GaAs).
The cases that work, they work also with the coefficients pre-calculated as explained in the last section.
Maybe the problem is in the double group issue mentioned above?.}


%======================================================

\section{Rotation of matrix elements}

The electron-phonon (e-ph) matrix elements are defined as
\begin{align}
M_{\alpha s \mathbf{q}}^{nm}(\mathbf{k}) = \bra{\psi_{n \mathbf{k}+\mathbf{q}}} 
  \partial^\alpha_{s \mathbf{q}} V(\mathbf{r},\mathbf{r'}) \ket{\psi_{m \mathbf{k}}}
\end{align}
where $\mathbf{q}$ is the phonon wavevector $\alpha=x,y,z$ and $s$ is the displaced atom.
We want to generate by rotation the elements for $\mathbf{q}$ and $\mathbf{k}$ 
in the full 1BZs (lists $\mathcal{M}_f^q$ and $\mathcal{M}_f^k$, respectively), from the elements 
calculated on the irreducible phonon wavector list $\mathbf{q}_s \in \mathcal{M}_i^q$. 

Our criterion for phases is that the rotated e-ph matrix elements have to be equal to 
those calculated for brakets with wavefunctions $\psi_{n \mathbf{k}}$ and 
$\psi_{n \mathbf{k}+\mathbf{q}}$ obtained by ``symlink'' rotation of $k$-vectors in 
the irreducible list $\mathcal{M}_i^k$. 

For simplicity, we start by taking a $\mathcal{S}$ with no TR and consider only the local 
part of the matrix elements in the analysis.

Take a $\mathbf{q} \in \mathcal{M}_f^q$ such that it is ``symlinked'' to 
$\mathbf{q}_s \in \mathcal{M}_i^q$ by $\mathcal{S}= \{ R|\tau \}$. Therefore, 
$R \mathbf{q}_s = \mathbf{q} + \mathbf{G}_q$, with $\mathbf{G}_q$ a reciprocal lattice vector.
The local contribution matrix elements are
\begin{align}
M_{\alpha s \mathbf{q}}^{nm}(\mathbf{k}) & = \bra{ \mathcal{SS}^{-1} \psi_{n \mathbf{k}+\mathbf{q}}}   
  \partial^\alpha_{s \mathbf{q}} V^{loc} \ket{\mathcal{SS}^{-1} \psi_{m \mathbf{k}}} = \\
 &= \bra{ \mathcal{S}^{-1} \psi_{n \mathbf{k}+\mathbf{q}} } 
   \mathcal{S}^{-1} \partial^\alpha_{s \mathbf{q}} V^{loc} \mathcal{S}
   \ket{ \mathcal{S}^{-1} \psi_{m \mathbf{k}} } 
\end{align}
where we have used $\mathcal{S}^\dagger = \mathcal{S}^{-1}$ and (see Haritz's Eq. 78):
\begin{align}
\partial^\alpha_{s \mathbf{q}} V^{loc}(\mathbf{r}) = -i \sum_{\mathbf{G}} (q_\alpha + G_\alpha) 
  \tilde v_s (\mathbf{q}+\mathbf{G}) e^{ i(\mathbf{q}+\mathbf{G}) \cdot \mathbf{r} }
\end{align} 
Here we see that we need to check the effect of $\mathcal{S}$ on the wavefunctions and 
on the potential derivative.

\textcolor{red}{NOTE: I think the steps above may be wrong if $\mathcal{S}$ has TR, because it would be antilinear. 
If we have spinors, the $\mathcal{S}$ has to be applied to the spins rotation in the potential derivative components, too.
On top of that, there is the double group issue.}



\subsection{Effect of wavefunction rotation}

We consider the transformed wavefunctions by $\mathcal{S}^{-1}$. 
These are calculated as in Eqs.~\ref{eq:rot_noTR1} and \ref{eq:rot_noTR2}, but they do not constitute a symlink.
In the following, we use the notation $\psi,u$ to denote transformations that correspond to a symlink and $\phi,v$ 
for those that do not. 
Explicitly:
\begin{align}
R^{-1} \mathbf{k}  & = \mathbf{k}_1 + \mathbf{G}_1 \\
R^{-1}(\mathbf{k}+\mathbf{q}) & = R^{-1}\mathbf{k}+R^{-1}\mathbf{q}  = 
  \mathbf{k}_1 + \mathbf{q}_s + \mathbf{G}_1 - R^{-1}\mathbf{G}_q 
\end{align}
where $\mathbf{k}_1 \in \mathcal{M}_f^k$ and
\begin{align}
\phi_{n \mathbf{k}_1} (\mathbf{r}) & \equiv \mathcal{S}^{-1} \psi_{n \mathbf{k}} (\mathbf{r})  \\
\phi_{n \mathbf{k}_1 + \mathbf{q}_s} (\mathbf{r}) & \equiv 
   \mathcal{S}^{-1} \psi_{n \mathbf{k}+\mathbf{q}} (\mathbf{r}) 
\end{align}
The Bloch wavefunctions corresponding to states with wavevectors $\mathbf{k}_1$ and $\mathbf{k}_1+\mathbf{q}_s$
can be calculated via symlink to states with $\mathbf{k}_1^s$ and $\mathbf{k}_2^s$, respectively, 
through transformations $\mathcal{S}_1$ and $\mathcal{S}_2$, respectively, i.e.,
\begin{align}
\mathcal{S}_1 \psi_{n_1 \mathbf{k}_1^s} (\mathbf{r}) & = \psi_{n_1 \mathbf{k}_1} (\mathbf{r}) \\ 
\mathcal{S}_2 \psi_{n_2 \mathbf{k}_2^s} (\mathbf{r}) & = \psi_{n_2 \mathbf{k}_1 + \mathbf{q}_s} (\mathbf{r}) 
\end{align}
such that there is a unitary transformation relation between these and 
$\phi_{n \mathbf{k}_1} (\mathbf{r})$ and $\phi_{n \mathbf{k}_1 + \mathbf{q}_s} (\mathbf{r})$, respectively.
Therefore, we can write
\begin{align}
\ket{\phi_{n  \mathbf{k}_1}} & = \sum_{n_1}  
   \braket{ \psi_{n_1 \mathbf{k}_1} | \phi_{n  \mathbf{k}_1} } 
   \ket{ \psi_{n_1 \mathbf{k}_1} } \\
\ket{ \phi_{n  \mathbf{k}_1 + \mathbf{q}_s} } & = \sum_{n_1}
   \braket{ \psi_{n_2 \mathbf{k}_1 + \mathbf{q}_s} | \phi_{n  \mathbf{k}_1 + \mathbf{q}_s} } 
   \ket{ \psi_{n_2 \mathbf{k}_1 + \mathbf{q}_s} }
\label{eq:unitaryrot}
\end{align}
The ``known'' and ``sought'' e-ph matrix elements are those calculated with $\psi$ functions, 
but by $\mathcal{S}^{-1}$ transformation we get the $\phi$ ones. Therefore, we have to ``undo'' the 
unitary transformation Eq.~\ref{eq:unitaryrot} (see section~\ref{sec:trios}).


\subsection{Effect of potential derivative rotation}

Since  $R\mathbf{q_s} = \mathbf{q} + \mathbf{G}_q$, the rotated local potential is  
\begin{align}
\mathcal{S}^{-1} & \partial^\alpha_{s \mathbf{q}} V^{loc}(\mathbf{r})\mathcal{S} =
   \partial^\alpha_{s \mathbf{q}} V^{loc}(\{R|\tau\} \mathbf{r}) = \\
&= -i \sum_{\mathbf{G}} (q_\alpha + G_\alpha)
   \tilde v_s (\mathbf{q}+\mathbf{G}) e^{ i(\mathbf{q}+\mathbf{G}) \cdot (R\mathbf{r}-\tau) } = \\
&= -i\sum_{\mathbf{G}} (R\mathbf{q_s} + \mathbf{G} - \mathbf{G}_q)_\alpha 
   \tilde v_s (R\mathbf{q_s} + \mathbf{G} - \mathbf{G}_q) 
   e^{ i( R\mathbf{q_s} + \mathbf{G} - \mathbf{G}_q ) \cdot R(\mathbf{r}+\tau) } = \\
&= -i\sum_{\mathbf{G}} (R\mathbf{q_s} + \mathbf{G})_\alpha 
   \tilde v_s (R\mathbf{q_s} + \mathbf{G}) 
   e^{ i( R \mathbf{q_s} + \mathbf{G}) \cdot R(\mathbf{r}+\tau) } 
\end{align}
where we have changed the sum over $G$ to $G-G_q$, and now we are going to change it to a sum over $R^{-1}G$.
We also write explicitly the Fourier components for the $s$-th atom, which is at position $\tau_s$ in the unit cell, 
according to Haritz's Eq.~(40) (square brakets):
\begin{align}
&= -i \sum_{\mathbf{G}} ( R(\mathbf{q_s} + R^{-1}\mathbf{G}) )_\alpha 
   \tilde v_s ( R(\mathbf{q_s} + R^{-1}\mathbf{G}) ) 
   e^{ i(\mathbf{q_s} + R^{-1}\mathbf{G}) \cdot ( \mathbf{r}-R^{-1}\tau)}  = \\
&= -i \sum_{\mathbf{G}} ( R(\mathbf{q_s} + \mathbf{G}) )_\alpha 
   \tilde v_s ( R(\mathbf{q_s} + \mathbf{G}) ) 
   e^{ i(\mathbf{q_s} + \mathbf{G}) \cdot ( \mathbf{r}+\tau)}  = \\
&= -i \sum_{\mathbf{G}} \sum_\beta R^{cart}_{\alpha\beta}(\mathbf{q_s} + \mathbf{G})_\beta 
   \Big[ e^{-i R(\mathbf{q_s} + \mathbf{G}) \cdot \tau_s } \frac{1}{\Omega}
   \int \mathrm{d}^3 \mathbf{r}' v_s(\mathbf{r}') e^{-i R(\mathbf{q_s} + \mathbf{G}) \cdot \mathbf{r}'}  \Big]
   e^{i (\mathbf{q_s} + \mathbf{G}) \cdot ( \mathbf{r}+\tau)}  = \\
&= -i \sum_\beta \sum_{\mathbf{G}} R^{cart}_{\alpha\beta}(\mathbf{q_s} + \mathbf{G})_\beta 
   e^{-i (\mathbf{q_s} + \mathbf{G}) \cdot (R^{-1} \tau_s - \tau) } 
   \Big[ \frac{1}{\Omega}
   \int \mathrm{d}^3 \mathbf{r}' v_s (\mathbf{r}') e^{-i (\mathbf{q_s} + \mathbf{G}) \cdot R^{-1}\mathbf{r}' }  \Big]
   e^{i (\mathbf{q_s} + \mathbf{G}) \cdot \mathbf{r} }
\end{align}
Since $v_s$ is the isolated atom pseudopotential, it is rotationally invariant, so doing the change 
$\mathbf{r}'' = R^{-1} \mathbf{r}'$ the integral term becomes
\begin{align}
 \frac{1}{\Omega} \int \mathrm{d}^3 \mathbf{r}''  v_s (\mathbf{r}'') e^{-i (\mathbf{q_s} + \mathbf{G}) \cdot \mathbf{r}'' } 
\end{align}
We take the $s'$-th atom related to $s$ by the symmetry operation $\mathcal{S}^{-1}$ as 
\begin{align}
\tau_{s'} = R^{-1} \tau_s - \tau + \mathbf{L}
\end{align} 
where $\mathbf{L}$ is a direct lattice vector. Since they are the same species, $v_s(\mathbf{r}) = v_{s'}(\mathbf{r})$.
Substituting above, we obtain $\tilde v_{s'} (\mathbf{q_s} + \mathbf{G})$ 
coefficients multiplied by the extra phase factor 
$e^{i (\mathbf{q_s} + \mathbf{G}) \cdot \mathbf{L} } = e^{i \mathbf{q_s} \cdot \mathbf{L} }$. Finally
\begin{align}
\mathcal{S}^{-1} & \partial^\alpha_{s \mathbf{q}} V^{loc}(\mathbf{r})\mathcal{S} = \\
 &= -i  \sum_\beta \sum_{\mathbf{G}} R^{cart}_{\alpha\beta}(\mathbf{q_s} + \mathbf{G})_\beta
   e^{i \mathbf{q_s} \cdot \mathbf{L} }  \tilde v_{s'} (\mathbf{q_s} + \mathbf{G}) 
   e^{i (\mathbf{q_s} + \mathbf{G}) \cdot \mathbf{r} } = \\
 &=    e^{i \mathbf{q_s} \cdot \mathbf{L} }   \sum_\beta R^{cart}_{\alpha\beta}  
    \partial^\beta_{s' \mathbf{q}_s} V^{loc}(\mathbf{r})
\end{align}

\textcolor{red}{It remains to see what happens if $\mathcal{S}$ involves TR and non-local terms. 
For the non-local term, I do not think any new factor will appear (the $e^{i \mathbf{q_s} \cdot \mathbf{L} }$
phase is associated to each atom, not each $\mathbf{r}$ or $\mathbf{r}'$ variable). 
For the TR case, since the potential derivative is a real operator, I do not think anything different will occur, 
at least in the no-spinor case.
Is that true?. }


\subsection{Full transformation}

Putting together the wavefunction and operator transformations we get
\begin{align}
M_{\alpha s \mathbf{q}}^{nm}(\mathbf{k}) & = 
  \braket{ \phi_{n \mathbf{k}_1+\mathbf{q}_s} 
   | \mathcal{S}^{-1} \partial^\alpha_{s \mathbf{q}} V^{loc} \mathcal{S} |
     \phi_{m \mathbf{k}_1} } = \\ 
&=  e^{i \mathbf{q_s} \cdot \mathbf{L} }  \sum_{n_1m_1} 
 \braket{ \phi_{n \mathbf{k}_1+\mathbf{q}_s} | \psi_{n_1 \mathbf{k}_1+\mathbf{q}_s} }
 \braket{ \psi_{m_1 \mathbf{k}_1} | \phi_{m \mathbf{k}_1} }
 \sum_\beta R^{cart}_{\alpha\beta} 
 \braket{ \psi_{n_1 \mathbf{k}_1+\mathbf{q}_s} 
         |  \partial^\beta_{s' \mathbf{q}_s} V^{loc}(\mathbf{r})  |
          \psi_{m_1 \mathbf{k}_1 } } 
\label{eq:ep_rotate}
\end{align}
where $\mathbf{q}$ is symlinked to $\mathbf{q}_s$ by $\mathcal{S} = \{R | \tau \}$ and 
the atoms are transformed as $\tau_s'=\mathcal{S}^{-1}\tau_s$.

In the case with spinors, we have to rotate the spin indices of the operator, too, as follows:
\begin{align}
\psi_{m\mathbf{k}}(\mathbf{r}) & = \mathcal{S} \phi_{m\mathbf{k}_1}(\mathbf{r}) = \\
&= D_{1/2}(\mathcal{S}) \Big( \phi_{m\mathbf{k}_1}(\mathcal{S}^{-1}\mathbf{r})^\uparrow \ket{\uparrow} + 
  \phi_{m\mathbf{k}_1}(\mathcal{S}^{-1}\mathbf{r})^\downarrow \ket{\downarrow}  \Big) \\
\psi_{m\mathbf{k}}(\mathbf{r})^\sigma &= \sum_{\sigma_1} D_{1/2}^{\sigma\sigma_1}(\mathcal{S}) i
  \phi_{m\mathbf{k}_1}^{\sigma_1}(\mathcal{S}^{-1}\mathbf{r})
\label{eq:ep_elem_full}
\end{align}
Thefore, the rotated matrix element for spins $\sigma,\pi$ is (already generalized for non-local):
\begin{align}
M_{\alpha s \mathbf{q}}^{nm,\sigma\pi}(\mathbf{k}) & = 
  \braket{ \phi_{n \mathbf{k}_1+\mathbf{q}_s}^\sigma
   | \mathcal{S}^{-1} \partial^\alpha_{s \mathbf{q}} V \mathcal{S} |
     \phi_{m \mathbf{k}_1}^\pi } = \\
& = \sum_{\sigma_1\pi_1} \Big( D_{1/2}^{\sigma\sigma_1}(\mathcal{S}) \Big)^* D_{1/2}^{\pi\pi_1}(\mathcal{S}) 
  \int \int \mathrm{d}^3\mathbf{r} \mathrm{d}^3\mathbf{r}' 
  \Big( \phi_{n \mathbf{k}_1+\mathbf{q}_s}^{\sigma_1} (\mathcal{S}^{-1}\mathbf{r} )  \Big)^*  
  \partial^\alpha_{s \mathbf{q}} V (\mathbf{r},\mathbf{r}') 
  \phi_{m \mathbf{k}_1}^{\pi_1} (\mathcal{S}^{-1}\mathbf{r}' ) = \\
& = \sum_{\sigma_1\pi_1} \Big( D_{1/2}^{\sigma\sigma_1}(\mathcal{S}) \Big)^* D_{1/2}^{\pi\pi_1}(\mathcal{S})   
M_{\alpha s \mathbf{q}}^{nm,\sigma_1\pi_1}(\mathbf{k}) 
\end{align}
where the last matrix element is Eq.~\ref{eq:ep_elem_full} evaluated for spin components $\sigma_1\pi_1$. 
If the $\mathcal{S}$ element contains TR, then we will need to multiply these $D_{1/2}$ rotation matrices by $i\sigma_y$.


%=====================================

\section{The unitary transformation}
\label{sec:trios}

In this section we calculate coefficients of type
\begin{align}
\braket{ \psi_{n \mathbf{k} } | \phi_{m \mathbf{k}} } = 
\braket{ \mathcal{S}_1 \psi_{n \mathbf{k}_s} | \mathcal{S}^{-1} \mathcal{S}_0 \psi_{m \mathbf{k}_s} } 
\label{eq:star_braket}
\end{align}
where $\psi$ and $\phi$ notations indicate wavefunctions obtained by symlink from a $k$-vector of 
the irreducible list $\mathcal{M}_i$ and wavefunctions obtained by a given symmetry $\mathcal{S}$.
By orthogonality of the wavefunctions, both sides of the braket belong to the same star, given 
by an irreducible $k$-vector $\mathbf{k}_s$. 

In the previous sections we have seen how to rotate matrix elements using this type of 
coefficients. However, if we use expressions such as Eq.~\ref{eq:ep_rotate}, we would need 
to calculate the coefficients for every $ \mathbf{k},  \mathbf{q}$ pair. 
Alternatively, we can use the property Eq.~\ref{eq:star_braket}, which implies that 
we need to calculate values only for the $\mathcal{M}_i$ list and only for the symmetry operations
that meet some invariance properties.


{\bf First, the basic case: no TR, no spinor, and only symmorphic symmetry elements.}
\begin{align}
\braket{ \psi_{n \mathbf{k} } | \phi_{m \mathbf{k}} } = 
\braket{ \psi_{n \mathbf{k}_s} | R_1^{-1} R^{-1} R_0 \psi_{m \mathbf{k}_s} }
\end{align}
The coefficients define a unitary rotation of the wavefunctions at point $\mathbf{k}_s$.
This will be non-zero for the trio of rotations $\tilde R = R_1^{-1} R^{-1} R_0$ such that 
$\tilde R \mathbf{k}_s = \mathbf{k}_s + \mathbf{G}_s$, i.e. $\tilde R \in G_{\mathbf{k}_s}$.
If $\tilde R = E$ (identity operator), the coefficients will be $e^{i\alpha} \delta_{nm}$ for 
some global phase $\alpha$. 
If $\tilde R \neq E$, band off-diagonal non-zero elements can appear. 
Therefore, we only need to evaluate
\begin{align}
\braket{ \psi_{n \mathbf{k}_s} | \tilde R \psi_{m \mathbf{k}_s} }, \quad \tilde R \in G_{\mathbf{k}_s}
\label{eq:sSs}
\end{align}
at the beginning of the calculation once and forall.

\textcolor{red}{For sure this is the theorem of Evarestov's book page 77, but I 
am a bit lost with their indices...}


{\bf Second, suppose TR is allowed in the elements of the trio, but still no spinor and only symmorphic.}
We will use the antilinearity property of the TR operator:
\begin{align}
\mathcal{T}^{-1} & = \mathcal{T}, \quad \mathcal{T}^2 = I, \quad \mathcal{T}R = R\mathcal{T} \\
\braket{ \mathcal{T} \psi | \mathcal{T} \varphi} & = \braket{\varphi | \psi}  \\
\braket{ \mathcal{T} \psi | \varphi} & = \braket{ \mathcal{T} \varphi | \psi } = \braket{ \psi | \mathcal{T} \varphi }^* 
\end{align}
We consider the trios $[ S_1 S S_0 ]$  and whether TR is present or not (1 or 0) in each:
\begin{itemize} 
\item $[000] \rightarrow$ $ \braket{ R_1 \psi | R^{-1} R_0 \varphi} = \braket{ \psi | R_1^{-1} R^{-1} R_0 \varphi} \equiv a$ 
\item $[010] \rightarrow$ $ \braket{ R_1 \psi | \mathcal{T} R^{-1} R_0 \varphi} = 
   \braket{ \psi | \mathcal{T} R_1^{-1} R^{-1} R_0 \varphi} \equiv b$ 
\item $[001] \rightarrow$ $ \braket{ R_1 \psi | R^{-1} \mathcal{T} R_0 \varphi} = 
   \braket{ \psi | \mathcal{T} R_1^{-1} R^{-1} R_0 \varphi} = b$ 
\item $[100] \rightarrow$ $ \braket{ \mathcal{T} R_1 \psi | R^{-1} R_0 \varphi} = 
   \braket{  R_1 \psi |  \mathcal{T} R^{-1} R_0 \varphi}^* = b^* $
\item $[011] \rightarrow$ $ \braket{ R_1 \psi | \mathcal{T} R^{-1} \mathcal{T} R_0 \varphi} = 
    \braket{ R_1 \psi | \mathcal{T}^2 R^{-1} R_0 \varphi} = a $
\item $[110]  \rightarrow$ $ \braket{ \mathcal{T} R_1 \psi | \mathcal{T} R^{-1} R_0 \varphi} = 
    \braket{  R_1 \psi |  \mathcal{T} \mathcal{T} R^{-1} R_0 \varphi}^* = a^* $
\item $[101]  \rightarrow$ $ \braket{ \mathcal{T} R_1 \psi | R^{-1} \mathcal{T} R_0 \varphi} = \dots = a^*$
\item $[111] \rightarrow$ $ \braket{ \mathcal{T} R_1 \psi | \mathcal{T} R^{-1}  \mathcal{T} R_0 \varphi} = \dots = b^*$
\end{itemize} 


{\bf Third, with spinors, symmorphic.} 
The TR operator is $\mathcal{T} = i\sigma_y \mathcal{K}$, where the Pauli matrix acts on the spin 
component and $\mathcal{K}$ is the complex conjugation acting on the wavefunctions. Therefore, 
\begin{align}
\mathcal{T}^{-1} & = - \mathcal{T} \quad \mathcal{T}^2 = -1 \quad \mathcal{T}R = R\mathcal{T} \\
\braket{ \mathcal{T} \psi | \mathcal{T} \varphi} & = \braket{\varphi | \psi}  \\
\braket{ \mathcal{T} \psi | \varphi} & = \braket{ \mathcal{T} \varphi |  \mathcal{T}^2 \psi} = 
   - \braket{ \mathcal{T} \varphi |  \psi} = - \braket{ \psi | \mathcal{T} \varphi }^*  \\
\end{align}
Again, we consider the trios $[ S_1 S S_0 ]$  and whether TR is present or not (1 or 0) in each:
\begin{itemize}
\item $[000] \rightarrow$ $ \braket{ R_1 \psi | R^{-1} R_0 \varphi} = \braket{ \psi | R_1^{-1} R^{-1} R_0 \varphi} \equiv a$
\item $[010] \rightarrow$ $ \braket{ R_1 \psi | (\mathcal{T}R)^{-1} R_0 \varphi} =
   -\braket{ \psi | \mathcal{T} R_1^{-1} R^{-1} R_0 \varphi} \equiv -b$
\item $[001] \rightarrow$ $ \braket{ R_1 \psi | R^{-1} \mathcal{T} R_0 \varphi} = b$
\item $[100] \rightarrow$ $ \braket{ \mathcal{T} R_1 \psi | R^{-1} R_0 \varphi} =
   -\braket{  R_1 \psi |  \mathcal{T} R^{-1} R_0 \varphi}^* = - b^* $
\item $[011] \rightarrow$ $ \braket{ R_1 \psi | (\mathcal{T}R)^{-1} \mathcal{T} R_0 \varphi} = 
    \braket{ \psi | R_1^{-1} R^{-1} \mathcal{T}^{-1} \mathcal{T} R_0 \varphi} = a $
\item $[110]  \rightarrow$ $ \braket{ \mathcal{T} R_1 \psi | (\mathcal{T} R)^{-1} R_0 \varphi} = 
    - \braket{  R_1 \psi |  \mathcal{T} \mathcal{T}^{-1} R^{-1} R_0 \varphi}^* = - a^* $
\item $[101]  \rightarrow$ $ \braket{ \mathcal{T} R_1 \psi | R^{-1} \mathcal{T} R_0 \varphi} = \dots = a^*$
\item $[111] \rightarrow$ $ \braket{ \mathcal{T} R_1 \psi | (\mathcal{T} R)^{-1}  \mathcal{T} R_0 \varphi} = \dots = - b^*$
\end{itemize}

\textcolor{red}{NOTE here I have not explicitely mentioned the double group issue in the inverse elements, 
which should be considered somehow...}


{\bf Fourth, we consider  non-symmorphic symmetries, but no TR and no spinors.} 
This is like the first case, in principle, but it may happen that the 
accumulated fractional translations  in the trio of elements, $\{R_1 | \tau_1\}, \{R^{-1} | \tau^i\}, \{R_0 | \tau_0\}$, 
differs by a lattice vector $\mathbf{L}$ from the translation in the total $\{ \tilde R | \tilde \tau \}$ from 
the list of group symmetry elements. This is discussed in Dresselhaus and Jorio, page 220. 
It gives rise to a phase $e^{i \mathbf{k}_s \cdot \mathbf{L} }$ to be added to the Eq.~\ref{eq:sSs} coefficients:
\begin{align}
\mathbf{L} = \tilde \tau + {\tilde R}^{-1} \tau_1 - R_0^{-1} \tau^i - \tau_0 
\end{align}
\textcolor{red}{TODO repasar notas... parece que funciona para Si, pero hay que tener cuidado con las transpuestas 
en los vectores del espacio real. Lo mejor es rotar siempre el $k$ y dejar las $\tau$ como están al calcular la fase.}

\textcolor{red}{DUDA hay algo que no entiendo: si añado la corrección Eq.~\ref{eq:Sinv_tau} para 
la operación "central" $\mathcal{S}^{-1}$ esto ya no funciona. 
No entiendo por qué hace falta en los $M_{mn}$ y no en e-ph...}


\section{Sakuma's dmn files}

The $D$ and $\tilde d$ matrices of Sakuma's paper, eqs. 15 and 17, respectively, are stored in the
Wannier90 .dmn file. They are written in blocks for all symmetry elements and
vectors $\mathbf{k}_s \in \mathcal{M}_{i}$.
Note, the method is not implemented yet in W90 for the spinor case. It does neither consider TR symmetry.

By definition, the $\tilde d$ matrices have elements
\begin{equation}
\tilde d_{mm'}(\mathcal{S},\mathbf{k}_s) = 
   \braket{\mathcal{S} \psi_{m\mathbf{k}_s} | \mathcal{S} \psi_{m'\mathbf{k}_s} }
\end{equation}
with the wavefunctions in the Bloch gauge,
for all $\mathcal{S}$ elements, not just the ones that leave $\mathbf{k}_s$ invariant.

On the other hand, $D$ is for wavefunctions in the Wannier gauge. They can be calculated for the
first iteration in W90 or the first trial, which is equivalent to using the wavefunctions
obtained from the $A_{mn}$ projections:
\begin{equation}
A_{mn}(\mathbf{k}_s) = \braket{\psi_{m\mathbf{k}_s} | g_n}
\end{equation}
where $g_n$ is a projection function. The trial function in the Wannier gauge is obtained from
\begin{equation}
\ket{\phi_{n\mathbf{k}_s}} = \sum_m A_{mn}(\mathbf{k}_s) \ket{\psi_{m\mathbf{k}_s}}
\end{equation}
where L\"owdin orthogonalization is applied as explained in Souza {\it et al}, PRB 65, 0350109:
\begin{equation}
\ket{\psi_{m\mathbf{k}_s}^{(0)} } = \sum_m (S^{-1/2})_{mn} \ket{\phi_{n\mathbf{k}_s}} = 
  \sum_m (AS^{-1/2})_{mn} \ket{\psi_{m\mathbf{k}_s}}
\end{equation}
Here, $AS^{-1/2}$, which is antiunitary, is obtained with SVD (zgesvd):
\begin{equation}
A = U \Sigma V^\dagger \quad \rightarrow \quad AS^{-1/2} = U 1 V^\dagger
\end{equation}
$A$ has dimensions $M \times N$, where $M$ is num\_bands\_intw and $N$ is num\_wann\_intw
(equal to the number of projections).
The $\psi_{m\mathbf{k}_s}^{(0)} (\mathbf{r})$ are used to calculate the $D$ matrix elements:
\begin{align}
D_{n'n}(\mathcal{S},\mathbf{k}_s) & =
  \braket{ \psi_{n'\mathbf{k}_s}^{(0)} | \mathcal{S} \psi_{n\mathbf{k}_s}^{(0)} } = \\
  & = \sum_{mm'} (U 1 V^\dagger)^*_{m'n'} (\mathcal{S} \mathbf{k}_s) (U 1 V^\dagger)_{mn} (\mathbf{k}_s)
   \braket{ \psi_{m'\mathcal{S} \mathbf{k}_s} | \mathcal{S}  \psi_{m  \mathbf{k}_s} }
\end{align}
The pw2w90 implementation calculates this integral explicitely with the $g_n$ functions.
Our method uses the $A_{mn}$ coefficients calculated previously in intw2w90.



\bibliography{intw_sym}

\end{document}



\bibliography{intw_sym}

\end{document}
