% !TeX encoding = UTF-8
% !TeX spellcheck = en_US
\documentclass[12pt,a4paper,onecolumn]{article}


\usepackage[utf8]{inputenc}
\usepackage[english]{babel}
%\usepackage[left=2.5cm,right=2.5cm,bottom=3cm]{geometry}
\usepackage[colorlinks=true,linkcolor=blue,citecolor=blue]{hyperref}

\usepackage{amsmath}
\usepackage{amsfonts}
\usepackage{amssymb}
\usepackage{graphicx}
%\usepackage{indentfirst}



%\usepackage{makeidx}
%\usepackage{wrapfig}
%\usepackage{float}
%\usepackage{caption}
%\usepackage{subcaption}
%\usepackage{subfigure}
%\usepackage{cite}
%\usepackage{pdfpages}
%\usepackage{fancyhdr}
%\pagestyle{fancy}

\title{Fourier transformation of the wave functions}
\author{}


%\lhead{Elektroi-fonoi elkarrekintza grafenoan}
%\rhead{Haritz Garai Marin}

\renewcommand{\vec}{\mathbf}

\begin{document}


\maketitle


\subsection*{Bloch's theorem}

The wave function can be written as
\begin{equation}\label{eq:bloch_1}
\psi_{\vec{k}}(\vec{r}) = e^{i\vec{k}\cdot\vec{r}} u_{\vec{k}}(\vec{r})
,\end{equation}
where $u_{\vec{k}}(\vec{r})$ has the periodicity of the crystal lattice. Thus,
\begin{equation}\label{eq:bloch_2}
\psi_{\vec{k}}(\vec{r}+\vec{R}_l) = e^{i\vec{k}\cdot\vec{R}_l} \psi_{\vec{k}}(\vec{r})
.\end{equation}


\subsection*{Fourier transform}

The Fourier transform of the periodic part of the wave function is
\begin{equation}
u_{\vec{k}}(\vec{G})
=
\frac{1}{\Omega}\int_\Omega u_{\vec{k}}(\vec{r})\, e^{-i\vec{G}\cdot\vec{r}}\, d\vec{r}
,\end{equation}
where $\Omega$ is the volume of the unit cell and $\vec{G}$ is a reciprocal lattice vector. Using equation \eqref{eq:bloch_1} it can be shown that
\begin{equation}
u_{\vec{k}}(\vec{G})
=
\psi_{\vec{k}}(\vec{k}+\vec{G})
.\end{equation}

Then, the wave function can be written in a Fourier series, as
\begin{align}\label{eq:wave function fourier series}
\psi_{\vec{k}}(\vec{r})
&=
e^{i\vec{k}\cdot\vec{r}} u_{\vec{k}}(\vec{r})
=
e^{i\vec{k}\cdot\vec{r}} \sum_{\vec{G}} u_{\vec{k}}(\vec{G}) e^{i\vec{G}\cdot\vec{r}} \nonumber\\
&=\sum_{\vec{G}} u_{\vec{k}}(\vec{G}) e^{i(\vec{k}+\vec{G})\cdot\vec{r}}\nonumber\\
&=\sum_{\vec{G}} \psi_{\vec{k}}(\vec{k}+\vec{G}) e^{i(\vec{k}+\vec{G})\cdot\vec{r}}
.\end{align}


\subsection*{Localized basis}

For a atomic localized basis set $\{\phi^\mu\}$, with $\mu=\{\vec{R}_{\tau},\alpha,l,m,\zeta\}$, to satisfy the Bloch's theorem we can define
\begin{equation}\label{eq:phi_k definition}
\phi_{\vec{k}}^{\mu}(\vec{r})
=
\sum_{\vec{R}_l} e^{i\vec{k}\cdot\vec{R}_l}\, \phi^\mu(\vec{r}-\vec{R}_l)
.\end{equation}
Now, the new basis set $\phi_{\vec{k}}^\mu$ will satisfy
\begin{equation}\label{eq:localized_1}
\phi_{\vec{k}}^\mu(\vec{r}+\vec{R}_l) = e^{i\vec{k}\cdot\vec{R}_l} \phi_{\vec{k}}^\mu(\vec{r})
,\end{equation}
and the wave functions can be written as
\begin{equation}
\psi_{n,\vec{k}}(\vec{r})
=
\sum_{\mu} c_{n,\vec{k}}^\mu\, \phi_{\vec{k}}^\mu(\vec{r})
.\end{equation}


\subsection*{Fourier transform of the localized basis}

Taking into account the equivalence between equations \eqref{eq:bloch_2} and \eqref{eq:localized_1}, we can write $\phi_{\vec{k}}^\mu$ in a Fourier series, in the same way that we have done with the wave function in equation \eqref{eq:wave function fourier series}:
\begin{align}
\phi_{\vec{k}}^\mu(\vec{r})
&=\sum_{\vec{G}} \phi_{\vec{k}}^\mu(\vec{k}+\vec{G}) e^{i(\vec{k}+\vec{G})\cdot\vec{r}}
.\end{align}

But now, using the definition of $\phi_{\vec{k}}^\mu$ in equation \eqref{eq:phi_k definition}, its Fourier transform is more complicated:
\begin{align}
\phi_{\vec{k}}^\mu(\vec{k}+\vec{G})
=
\sum_{\vec{R}_l} e^{i\vec{k}\cdot\vec{R}_l}\, e^{-i(\vec{k}+\vec{G})\cdot\vec{R}_l}\, \phi^\mu(\vec{k}+\vec{G})
.\end{align}

%\bibliographystyle{abbrv}
%\bibliographystyle{plain}
%\bibliographystyle{apacite}
%\bibliographystyle{unsrt}
%\bibliography{bibliografi.bib}


\end{document}
